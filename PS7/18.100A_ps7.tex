\documentclass{article}
\usepackage[utf8]{inputenc}
\usepackage[english]{babel}
\usepackage[]{amsthm} %lets us use \begin{proof}
\usepackage[]{amssymb} %gives us the character \varnothing
\usepackage[]{amsmath}
\usepackage[parfill]{parskip} %avoid indent when skipping lines
\usepackage[toc,page]{appendix}
\usepackage{mathtools}
\usepackage{hyperref}
\hypersetup{
	colorlinks=true,
	linkcolor=blue,
	filecolor=magenta,      
	urlcolor=cyan,
	pdftitle={18.100A-ps7},
	pdfpagemode=FullScreen,
}
%\urlstyle{same}
\newcommand{\R}{\mathbb{R}} %the real numbers
\newcommand{\N}{\mathbb{N}} %the natural numbers
\newcommand{\M}{\mathcal{M}} %set of all Lebesgue-measurable sets
\newcommand{\Q}{\mathbb{Q}} % the rational numbers

\title{18.100A Assignment 7}
\author{Octavio Vega}
\date\today

\begin{document}
\maketitle
	
\section*{Problem 1}
\begin{proof}
	Since $\sum_n a_n$ and $\sum_n b_n$ converge absolutely, suppose that $\sum_n |a_n| < M$ and $\sum_n |b_n| < N$. Then
	\begin{align}
		\sum_{n=0}^{m}|c_n| &= \sum_{n=0}^m \left|\sum_{k=0}^n a_k b_{n-k}\right| \\
		& \leq \sum_{n=0}^m \sum_{k=0}^n |a_k b_{n-k}| \\
		&= |a_0 b_0| + (|a_0 b_1| + |a_1 b_0|) + \cdots + \nonumber \\&(|a_0 b_m | + |a_1 b_{m-1}| + \cdots + |a_m b_0|) \\
		&= \sum_{n=0}^m |a_n| \sum_{k=0}^{m-n}|b_k| \\
		&< MN.
	\end{align}
	Thus $\sum_n |c_n|$ is bounded above and monotone, so it converges.
\end{proof}
%%%%%%%%%%%%%%%%%%%%%%%%%%%%%%%%%%%%%%%%%%%%%%%%%%%%%%%%%%%%%%%%%%%%%%%%%%%%%%%%%%%%%%%%%%%%%%%%%%%%%%%%%%%%%%%%%%%%%%%%%%%%%%%%%%%%%%%%%%%%
\section*{Problem 2}
\subsection*{(a)}
Let $a_n = 2^n x^n$. Then $\left|\frac{a_{n+1}}{a_n}\right| = \left|\frac{2^{n+1}x^{n+1}}{2^n x^n}\right| = 2 |x|$.

By the ratio test, we must have
\begin{equation}
	L = \lim_{n \to \infty} 2 |x| < 1.
\end{equation}
Thus, $\sum_{n=0}^{\infty} 2^n x^n$ converges for all $|x| < \frac{1}{2}$.

\subsection*{(b)}
TODO TODO

\end{document}