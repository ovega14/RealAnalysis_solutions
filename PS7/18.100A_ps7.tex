\documentclass{article}
\usepackage[utf8]{inputenc}
\usepackage[english]{babel}
\usepackage[]{amsthm} %lets us use \begin{proof}
\usepackage[]{amssymb} %gives us the character \varnothing
\usepackage[]{amsmath}
\usepackage[parfill]{parskip} %avoid indent when skipping lines
\usepackage[toc,page]{appendix}
\usepackage{mathtools}
\usepackage{hyperref}
\hypersetup{
	colorlinks=true,
	linkcolor=blue,
	filecolor=magenta,      
	urlcolor=cyan,
	pdftitle={18.100A-ps7},
	pdfpagemode=FullScreen,
}
%\urlstyle{same}
\newcommand{\R}{\mathbb{R}} %the real numbers
\newcommand{\N}{\mathbb{N}} %the natural numbers
\newcommand{\M}{\mathcal{M}} %set of all Lebesgue-measurable sets
\newcommand{\Q}{\mathbb{Q}} % the rational numbers

\title{18.100A Assignment 7}
\author{Octavio Vega}
\date\today

\begin{document}
\maketitle
	
\section*{Problem 1}
\begin{proof}
	Since $\sum_n a_n$ and $\sum_n b_n$ converge absolutely, suppose that $\sum_n |a_n| < M$ and $\sum_n |b_n| < N$. Then
	\begin{align}
		\sum_{n=0}^{m}|c_n| &= \sum_{n=0}^m \left|\sum_{k=0}^n a_k b_{n-k}\right| \\
		& \leq \sum_{n=0}^m \sum_{k=0}^n |a_k b_{n-k}| \\
		&= |a_0 b_0| + (|a_0 b_1| + |a_1 b_0|) + \cdots + \nonumber \\&(|a_0 b_m | + |a_1 b_{m-1}| + \cdots + |a_m b_0|) \\
		&= \sum_{n=0}^m |a_n| \sum_{k=0}^{m-n}|b_k| \\
		&< MN.
	\end{align}
	Thus $\sum_n |c_n|$ is bounded above and monotone, so it converges.
\end{proof}
%%%%%%%%%%%%%%%%%%%%%%%%%%%%%%%%%%%%%%%%%%%%%%%%%%%%%%%%%%%%%%%%%%%%%%%%%%%%%%%%%%%%%%%%%%%%%%%%%%%%%%%%%%%%%%%%%%%%%%%%%%%%%%%%%%%%%%%%%%%%
\section*{Problem 2}
\subsection*{(a)}
Let $a_n = 2^n x^n$. Then $\left|\frac{a_{n+1}}{a_n}\right| = \left|\frac{2^{n+1}x^{n+1}}{2^n x^n}\right| = 2 |x|$.

By the ratio test, we must have
\begin{equation}
	L = \lim_{n \to \infty} 2 |x| < 1.
\end{equation}
Thus, $\sum_{n=0}^{\infty} 2^n x^n$ converges for all $|x| < \frac{1}{2}$.

\subsection*{(b)}
We have $a_n = n x^n$, so $\left|\frac{a_{n+1}}{a_n}\right| = \left|\frac{(n+1)x^{n+1}}{nx^n}\right| = \frac{n+1}{n}|x|$.

Thus, we require
\begin{equation}
	\lim_{n \to \infty} \frac{n+1}{n}|x| < 1.
\end{equation}
Therefore, $\sum_n n x^n$ converges for all $|x| < 1$. 

\subsection*{(c)}
Proceeding with the ratio test, we have
\begin{equation}
	\left|\frac{a_{n+1}}{a_n}\right| = \left|\frac{(x-10)^{n+1} (2n)!}{(2n+2)!(x-10)^n}\right| = \left|\frac{x-10}{(2n+2)(2n+1)}\right|
\end{equation}
Then, we require
\begin{equation}
	\lim_{n \to \infty} \left|\frac{x-10}{4n^2 + 6n + 2}\right| = 0 < 1,
\end{equation}
which is always satisfied. Thus, $\sum_n \frac{1}{(2n)!}(x-10)^n$ converges $\forall x \in \R$.

\subsection*{(d)}
Letting $a_n = n! x^n$, we have 
\begin{equation}
	\left|\frac{a_{n+1}}{a_n}\right| = \left|\frac{(n+1)!x^{n+1}}{n! x^n}\right| = (n+1)|x|.
\end{equation}
Thus we must have
\begin{equation}
	\lim_{n \to \infty} (n+1)|x| < 1,
\end{equation}
which is only satisfied for $x=0$. Thus, $\sum_n n! x^n$ converges only for $x=0$.
%%%%%%%%%%%%%%%%%%%%%%%%%%%%%%%%%%%%%%%%%%%%%%%%%%%%%%%%%%%%%%%%%%%%%%%%%%%%%%%%%%%%%%%%%%%%%%%%%%%%%%%%%%%%%%%%%%%%%%%%%%%%%%%%%%%%%%%%%%%%
\section*{Problem 3}
\begin{proof}
	(i) Let $z_n = \max\{|x_n|, |y_n|\}$ for each $n \in \N$. Then
	\begin{equation} \label{3.1}
		|x_n y_n| = |x_n||y_n| \leq |x_n||z_n| \leq |z_n|^2.
	\end{equation}
	But we assumed that both $\sum_n |x_n|^2$ and $\sum_n |y_n|^2$ converge, so $\sum_n |z_n|^2$ converges. Thus, by \eqref{3.1} we see that $\sum_n |x_n y_n|$ converges by comparison.
	
	Therefore, $\sum_n x_n y_n$ converges absolutely.
	
	(ii) By the triangle inequality for infinite series, we have
	\begin{equation}
		\left|\sum_{n=1}^{\infty} x_n y_n \right| \leq \sum_{n=1}^{\infty}|x_n y_n|.
	\end{equation}
	Let $N \in \N$ and let $A = \sqrt{x_1^2 + \cdots + x_N^2}$ and $B = \sqrt{y_1^2 + \cdots + y_N^2}$. By the arithmetic mean - geometric mean inequality, for each $n \in \N$ we have
	\begin{equation}
		\sqrt{\frac{x_n^2 y_n^2}{A^2 B^2}} \leq \frac{1}{2}\left(\frac{x_n^2}{A^2} + \frac{y_n^2}{B^2}\right).
	\end{equation}
	Summing over all $k = 1, ..., N$,
	\begin{equation} \label{3.2}
		\sum_{n=1}^N \frac{x_n y_n}{AB} \leq \sum_{n=1}^N \frac{1}{2}\left(\frac{x_n^2}{A^2} + \frac{y_n^2}{B^2}\right) = 1
	\end{equation}
	Multiplying both sides of \eqref{3.2} by $AB$, we get
	\begin{equation}
		\sum_{n=1}^N x_n y_n \leq AB = \left(\sum_{n=1}^N x_n^2\right)^\frac{1}{2}\left(\sum_{n=1}^N y_n^2\right)^\frac{1}{2}.
	\end{equation}
	Letting $N \to \infty$, since limits respect inequalities we arrive at
	\begin{equation}
		\sum_{n=1}^{\infty}|x_n y_n| \leq \left(\sum_{n=1}^{\infty}x_n^2\right)^\frac{1}{2}\left(\sum_{n=1}^{\infty}y_n^2\right)^\frac{1}{2},
	\end{equation}
	as desired.
\end{proof}
%%%%%%%%%%%%%%%%%%%%%%%%%%%%%%%%%%%%%%%%%%%%%%%%%%%%%%%%%%%%%%%%%%%%%%%%%%%%%%%%%%%%%%%%%%%%%%%%%%%%%%%%%%%%%%%%%%%%%%%%%%%%%%%%%%%%%%%%%%%%
\section*{Problem 4}
\begin{proof}
	Let $x \in \R$ and let $\epsilon > 0$. 
	
	Consider the set $(x - \epsilon, x + \epsilon)$. We already showed that $\forall x,\;y \in \R$ with $x < y$, $\exists z \in \R \backslash \Q$ such that $x < z < y$.
	
	Thus $\forall x \in \R$ and for every $\epsilon > 0$,
	\begin{equation}
		(x - \epsilon, x + \epsilon) \cap (\R \backslash \Q) \backslash \{x\} \neq \varnothing.
	\end{equation}
	Therefore, every real number is a cluster point of the irrationals.
\end{proof}
%%%%%%%%%%%%%%%%%%%%%%%%%%%%%%%%%%%%%%%%%%%%%%%%%%%%%%%%%%%%%%%%%%%%%%%%%%%%%%%%%%%%%%%%%%%%%%%%%%%%%%%%%%%%%%%%%%%%%%%%%%%%%%%%%%%%%%%%%%%%
\section*{Problem 5}
\begin{proof}
	Since $c$ is a cluster point of $S$, then $\exists$ a sequence $\{y_k\}_k$ of elements in $S \backslash \{c\}$ such that $y_k \to c$.
	
	Also, since $f$ is bounded, then $\exists$ $B \geq 0$ such that $\forall x \in S$, $|f(x)| \leq B$. Then the sequence $\{f(y_k)\}_k$ is bounded. By the Bolzano-Weierstrass theorem, $\exists$ a convergent subsequence $\{f(y_{k_n})\}_n$. Simply taking $x_n = y_{k_n}$ for each $n \in \N$, we see that $\{f(x_n)\}_n$ converges, as desired.
\end{proof}
\end{document}