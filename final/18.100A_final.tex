\documentclass{article}
\usepackage[utf8]{inputenc}
\usepackage[english]{babel}
\usepackage[]{amsthm} %lets us use \begin{proof}
\usepackage[]{amssymb} %gives us the character \varnothing
\usepackage[]{amsmath}
\usepackage[parfill]{parskip} %avoid indent when skipping lines
\usepackage[toc,page]{appendix}
\usepackage{mathtools}
\usepackage{hyperref}
\hypersetup{
	colorlinks=true,
	linkcolor=blue,
	filecolor=magenta,      
	urlcolor=cyan,
	pdftitle={18.100A-final},
	pdfpagemode=FullScreen,
}
%\urlstyle{same}
\newcommand{\R}{\mathbb{R}} %the real numbers
\newcommand{\N}{\mathbb{N}} %the natural numbers
\newcommand{\M}{\mathcal{M}} %set of all Lebesgue-measurable sets
\newcommand{\Q}{\mathbb{Q}} % the rational numbers

\title{18.100A Final}
\author{Octavio Vega}
\date\today

\begin{document}
\maketitle
	
\section*{Problem 1}
We complete the following \textbf{negations}:
\subsection*{(i)}
Let $S \subset \R$. A function $f: S \to \R$ is \textbf{not continuous} at $c \in S$ if $\exists \epsilon_0 > 0$ such that $\forall \delta > 0$ if $|x - c| < \delta$, $|f(x) - f(c)| \geq \epsilon_0$.

\subsection*{(ii)}
Let $S \subset \R$. A function $f: S \to \R$ is \textbf{not uniformly continuous} on $S$ if $\exists x_0 \in S$ such that $\forall \delta > 0$ $\exists \epsilon_0 > 0$ such that if $|x_0 - x| < \delta$, then $|f(x) - f(x_0)| \geq \epsilon_0$.

\subsection*{(iii)}
Let $S \subset \R$. A sequence of functions $f_n: S \to \R$ \textbf{does not converge uniformly} to $f: S \to \R$ if $\exists \epsilon_0 > 0$ such that $\forall M \in \N$ $\exists n \geq M$ and $x \in S$ such that $|f_n(x) - f(x)| \geq \epsilon_0$.
%%%%%%%%%%%%%%%%%%%%%%%%%%%%%%%%%%%%%%%%%%%%%%%%%%%%%%%%%%%%%%%%%%%%%%%%%%%%%%%%%%%%%%%%%%%%%%%%%%%%%%%%%%%%%%%%%%%%%%%%%%%%%%%%%%%%%%%%%%%%
\section*{Problem 2}
\subsection*{(a)}
\subsubsection*{(i)}
A continunous function on $(0, 1)$ with neither a global minimum or maximum:

Let $f(x) = 1$ $\forall x \in (0, 1)$. Then $\forall x, y \in (0, 1)$, $f(x) = f(y)$ so $f$ has no absolute maximum or minimum, and $f$ is constant and therefore continuous.

\subsubsection*{(ii)}
A function on $[0, 1]$ with absolute minimum at $0$, absolute maximum at $1$, and such that $\exists y \in (f(0), f(1))$ not in the range of $f$:

Define $f$ via
\begin{align}
	f(x) := 
	\begin{cases}
		x, \quad &x \in \left(0, \frac{1}{2}\right) \cap \left(\frac{1}{2}, 1\right] \\
		-1, &x = 0 \\
		-\frac{1}{2}, &x = \frac{1}{2}.
	\end{cases}
\end{align}
Then, for example, $-\frac{3}{4} \in (-1, 1) = \left(f(0), f(1)\right)$, but $-\frac{3}{4}$ is not in the range of $f$. Also, $f$ has an absolute minimum and maximum at 0 and 1, respectively. 

\subsection*{(b)}
\begin{proof}
	Let $\epsilon > 0$. Since $f$ is continuous, then $\exists \delta_0 > 0$ such that if $|x - c| < \delta_0$ then $|f(x) - f(c)| < \frac{\epsilon}{2}$. Similarly, since $g$ is continuous, then $\exists \delta_1 > 0$ such that if $|x - c| < \delta_1$ then $|g(x) - g(c)| < \frac{\epsilon}{2}$. 
	
	Choose $\delta_0, \delta_1$ such that $|f(x)| + |g(c)| < 2$, and let $\delta = \min\{\delta_0, \delta_1\}$. Then if $|x - c| < \delta$, we have
	\begin{align}
		|f(x)g(x) - f(c)g(c)| &= |f(x)g(x) - f(x)g(c) + f(x)g(c) - f(c)g(c)| \\
		&\leq |f(x)||g(x) - g(c)| + |g(c)||f(x) - f(c) \\
		&< |f(x)|\frac{\epsilon}{2} + |g(c)|\frac{\epsilon}{2} \\
		&< \epsilon.
	\end{align} 
	Therefore, the product $fg$ is continuous at $c$. 
\end{proof}
%%%%%%%%%%%%%%%%%%%%%%%%%%%%%%%%%%%%%%%%%%%%%%%%%%%%%%%%%%%%%%%%%%%%%%%%%%%%%%%%%%%%%%%%%%%%%%%%%%%%%%%%%%%%%%%%%%%%%%%%%%%%%%%%%%%%%%%%%%%%
\section*{Problem 3}
\subsection*{(a)}
\begin{proof}
	Let $\{x_n\}_n$ be a sequence of elements in $[a, b]$. Choose $B = \max\{|a|, |b|\}$. Then $\forall n \in \N$, $|x_n| \leq B$, so the sequence $\{x_n\}_n$ is bounded.
	
	By the Bolzano-Weierstrass theorem, $\exists$ a subsequence $\{x_{n_k}\}_k \subset [a, b]$ that converges, i.e. such that $x_{n_k} \to x$ as $k \to \infty$ for some $x \in \R$. Since $[a, b]$ is closed, then $x \in [a, b]$.
	
	Therefore $[a, b]$ is compact.
\end{proof}

\subsection*{(b)}
\begin{proof}
	Let $\{x_n\}_n$ be a sequence of elements in $[a, b]$. Since $[a, b]$ is compact, $\exists \{x_{n_k}\}_k \subset [a, b]$ such that $x_{n_k} \to x$ for some $x \in [a, b]$. 
	
	Then $\{f(x_{n_k})\}_k$ is a subsequence in $[a, b]$, and since $f$ is continuous we have
	\begin{align}
		\lim\limits_{k \to \infty}f\left(x_{n_k}\right) &= f\left(\lim\limits_{k \to \infty} x_{n_k}\right) \\
		&= f(x) \in f \left([a, b]\right),
	\end{align}
	so the subsequence $\{f_{n_k}\}_k$ of $\{f_n\}_n$ converges in $[a, b]$. 
	
	Therefore $f\left([a, b]\right)$ is compact.
\end{proof}
%%%%%%%%%%%%%%%%%%%%%%%%%%%%%%%%%%%%%%%%%%%%%%%%%%%%%%%%%%%%%%%%%%%%%%%%%%%%%%%%%%%%%%%%%%%%%%%%%%%%%%%%%%%%%%%%%%%%%%%%%%%%%%%%%%%%%%%%%%%%
\section*{Problem 4}
\subsection*{(a)}
\subsubsection*{(i)}
\begin{proof}
	Since $f$ is differentiable at $c$. then the limit
	\begin{equation}
		f'(c) = \lim\limits_{x \to c} \frac{f(x) - f(c)}{x - c}
	\end{equation}
	exists. Then we have
	\begin{align}
		\lim\limits_{x \to c} \left(f(x) - f(c)\right) &= \lim\limits_{x \to c} \left[\left(\frac{f(x) - f(c)}{x-c}\right)(x - c)\right] \\
		&= \lim\limits_{x \to c} \left(\frac{f(x) - f(c)}{x - c}\right) \cdot \lim\limits_{x \to c} (x - c) \\
		&= f'(c) \lim\limits_{x \to c} (x - c) \\
		&= 0.
	\end{align}
	Therefore $\lim\limits_{x \to c} f(x) = f(c)$, so $f$ is continuous at $c$.
\end{proof}

\subsubsection*{(ii)} 
Consider the following example disproving the converse of part \textbf{(i)}:

$f(x) = |x|$ is continuous but not differentiable at $0$. Checking this is simple: we simply compute the left and right limits of the difference quotient and find that they do not agree, signifying that the limit (derivative) does not exist at $0$. 

\subsection*{(b)}
\begin{proof}
	Using the limit definition of the derivative, we have
	\begin{align}
		f'(0) &= \lim\limits_{x \to 0} \frac{x^2 \sin\left(\frac{1}{x}\right)}{x} \\
		&= \lim\limits_{x \to 0} \frac{\sin\left(\frac{1}{x}\right)}{\frac{1}{x}} \\
		&= 0,
	\end{align}
	so the limit exists.
	
	Therefore $f$ is differentiable at $0$. 
\end{proof}
%%%%%%%%%%%%%%%%%%%%%%%%%%%%%%%%%%%%%%%%%%%%%%%%%%%%%%%%%%%%%%%%%%%%%%%%%%%%%%%%%%%%%%%%%%%%%%%%%%%%%%%%%%%%%%%%%%%%%%%%%%%%%%%%%%%%%%%%%%%%
\section*{Problem 5}
\subsection*{(a)}
\begin{proof}
	Let $f(x) = e^x$. By Taylor's theorem, $\forall x \in [-R, R]$ $\exists c \in (0, x)$ such that
	\begin{equation}
		f(x) = \sum_{k=0}^n \frac{1}{k!}f^{(k)}(0)x^k + \frac{f^{(n+1)}(c)}{(n+1)!}x^{n+1}
	\end{equation},
	which is equivalent to
	\begin{equation}
		e^x - \sum_{k=0}^n \frac{x^k}{k!} = \frac{e^c}{(n+1)!}x^{n+1}
	\end{equation}
	But $|x| \leq R$ and $c < R$, therefore
	\begin{equation}
		e^x - \sum_{k=0}^n \frac{x^k}{k!} \leq \frac{e^R}{(n+1)!}R^{n+1},
	\end{equation}
	as desired.
\end{proof}

\subsection*{(b)}
\begin{proof}
	Using integration by parts, we have
	\begin{align}
		\int_c ^x f''(t) (x - t) \mathrm{d}t &= (x - t) f'(t) \big|_c ^x + \int_c ^x f'(t) \mathrm{d}t \\
		&= -(x - c) f'(c) + f(x) - f(c),
	\end{align}
	i.e. 
	\begin{equation}
		f(x) = f(c) + f'(c) (x - c) + \int_c ^x f''(t) (x - c) \mathrm{d}t,
	\end{equation}
	as desired.
\end{proof}
%%%%%%%%%%%%%%%%%%%%%%%%%%%%%%%%%%%%%%%%%%%%%%%%%%%%%%%%%%%%%%%%%%%%%%%%%%%%%%%%%%%%%%%%%%%%%%%%%%%%%%%%%%%%%%%%%%%%%%%%%%%%%%%%%%%%%%%%%%%%


\end{document}