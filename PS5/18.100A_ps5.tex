\documentclass{article}
\usepackage[utf8]{inputenc}
\usepackage[english]{babel}
\usepackage[]{amsthm} %lets us use \begin{proof}
\usepackage[]{amssymb} %gives us the character \varnothing
\usepackage[]{amsmath}
\usepackage[parfill]{parskip} %avoid indent when skipping lines
\usepackage{mathtools}
\usepackage{hyperref}
\hypersetup{
	colorlinks=true,
	linkcolor=blue,
	filecolor=magenta,      
	urlcolor=cyan,
	pdftitle={18.100A-ps5},
	pdfpagemode=FullScreen,
}
%\urlstyle{same}

\title{18.100A Assignment 5}
\author{Octavio Vega}
\date\today

\begin{document}
\maketitle

\section*{Problem 1}
\begin{proof}
	We have that
	\begin{equation}
		L = \lim\limits_{n\to\infty} \frac{|x_{n+1}-x|}{|x_n-x|} < 1.
	\end{equation}
	Then for every $\epsilon > 0$, $\exists N \in \mathbb{N}$ such that $\forall n \geq N$, 
	\begin{equation}
		\left|\frac{x_{n+1}-x}{x_n - x}\right| - 1 < \epsilon.
	\end{equation}
	Rearranging gives
	\begin{equation}
		|x_{n+1}-x| < (1+\epsilon)|x_n - x|.
	\end{equation}
	By taking $\epsilon$ to be arbitrarily small, we have, $\forall n \geq N$,
	\begin{equation}
		\xRightarrow{\epsilon \to 0} |x_{n+1}-x| < |x_n-x|.
	\end{equation}
	Define $y_n := |x_n - x|$. Then $\forall n \geq N$, 
	\begin{equation}
		0 \leq y_{n+1} < y_n,
	\end{equation}
	so $\{y_n\}_n$ is a decreasing sequence bounded below by $0$. Hence,
	\begin{align}
		&\implies y_n \rightarrow 0 \\
		&\implies |x_n - x| \rightarrow 0 \\
		&\implies x_n \rightarrow x.
	\end{align}
	Therefore, $\{x_n\}_n$ converges to $x$.
\end{proof}
%%%%%%%%%%%%%%%%%%%%%%%%%%%%%%%%%%%%%%%%%%%%%%%%%%%%%%%%%%%%%%%%%%%%%%%%%%%%%%%%%%%%%%%%%%%%%%%%%%%%%%%%%%%%%%%%%%%%%%%%%%%%%%%%%%%%%%%%%%%%
\section*{Problem 2}
\subsection*{(a)}
Let $x_n = \frac{(-1)^n}{n}$. Then $\forall n \in \mathbb{N}$,
\begin{equation}
	-\frac{1}{n} \leq x_n \leq \frac{1}{n}.
\end{equation}
Allowing $n \to \infty$ on all sides of the inequality gives
\begin{equation}
	0=\lim\limits_{n\to\infty} \left(-\frac{1}{n}\right) \leq \lim_{n \to \infty} x_n \leq \lim_{n\to\infty} \left(\frac{1}{n}\right)=0.
\end{equation}
So by the Squeeze Theorem, $\lim_{n\to\infty} x_n = 0$. Finally, by the theorem from lecture 9, we conclude that $\lim\inf{x_n}=\lim\sup{x_n}=0$.

\subsection*{(b)}
Let $x_n = (-1)^n\frac{(n-1)}{n}$. Define
\begin{equation}
	a_n := \sup\{x_k \;|\; k\geq n\} \textrm{, and } b_n := \sup\{x_k \;|\; k\geq n\}.
\end{equation}
Then $\forall n \in \mathbb{N}$, we have 
\begin{align}
	|x_n| &= \left|\frac{n-1}{n}\right| \cdot |(-1)^n| \\
	&= \frac{n-1}{n} \\
	&= 1-\frac{1}{n} \\
	&\leq 1.
\end{align}
Thus, $x_n$ is bounded and $-1\leq x_n \leq 1$. 

Let $n_k=2k$ and $m_k=2k-1$ for $k\in\mathbb{N}$. Then we construct two subsequences $\{x_{n_k}\}_k$ and $\{x_{m_k}\}_k$ of $\{x_n\}_n$ via
\begin{equation}
	x_{n_k} := \frac{2k-1}{2k}(-1)^{2k} = \frac{2k-1}{2k},
\end{equation}
and
\begin{equation}
	x_{m_k} := \frac{2k-2}{2k-1}(-1)^{2k-1} = \frac{2-2k}{2k-1}.
\end{equation}
Taking the limit of the first subsequence gives 
\begin{equation}
	\lim_{k \to \infty} x_{n_k} = \lim_{k \to\infty} \left(1-\frac{1}{2k}\right) = 1,
\end{equation}
and for the second subsequence we get
\begin{equation}
	\lim_{k\to\infty} x_{m_k} = \lim_{k \to \infty}\left(\frac{1}{2k-1}-1\right)=-1.
\end{equation}
Thus, $\{x_{n_k}\}_k$ converges to $1$, and $\{x_{m_k}\}_k$ converges to $-1$. Then \\$\sup\{x_k \;|\; k\geq n\} \geq 1$. But $-1\leq x_n \leq 1$, so it must also be true that
\begin{equation}
	\sup\{x_k \;|\; k\geq n\} \leq \sup x_n = 1.
\end{equation}
Thus, $\sup\{x_k \;|\; k\geq n\}=1$, and we can conclude that $\lim\sup x_n =1$.

Similarly, $\inf\{x_k \;|\; k\geq n\}\leq -1$, but $-1\leq x_n \leq 1$, so we must have
\begin{equation}
	\inf \{x_k \;|\; k\geq n\} \geq \inf x_n = -1,
\end{equation}
so $\inf \{x_k \;|\; k\geq n\}=1$. Therefore, we conclude also that $\lim\inf x_n = -1$.
\end{document}