\documentclass{article}
\usepackage[utf8]{inputenc}
\usepackage[english]{babel}
\usepackage[]{amsthm} %lets us use \begin{proof}
\usepackage[]{amssymb} %gives us the character \varnothing
\usepackage[]{amsmath}
\usepackage[parfill]{parskip} %avoid indent when skipping lines
\usepackage[toc,page]{appendix}
\usepackage{mathtools}
\usepackage{hyperref}
\hypersetup{
	colorlinks=true,
	linkcolor=blue,
	filecolor=magenta,      
	urlcolor=cyan,
	pdftitle={18.100A-midterm},
	pdfpagemode=FullScreen,
}
%\urlstyle{same}
\newcommand{\R}{\mathbb{R}} %the real numbers
\newcommand{\N}{\mathbb{N}} %the natural numbers
\newcommand{\M}{\mathcal{M}} %set of all Lebesgue-measurable sets
\newcommand{\Q}{\mathbb{Q}} % the rational numbers

\title{18.100A Midterm}
\author{Octavio Vega}
\date\today

\begin{document}
\maketitle

\section*{Problem 1}
\subsection*{(a)}
\begin{proof}
Let $x \in f^{-1}(C \cap D)$. Then 
\begin{align}
	&\implies f(x) \in C \cap D \\
	&\implies f(x) \in C \textrm{ and } f(x) \in D \\
	&\implies x \in f^{-1}(C) \textrm{ and } x \in f^{-1}(D) \\
	&\implies x \in f^{-1}(C) \cap f^{-1}(D).
\end{align}
Thus, 
\begin{equation}\label{1a-1}
	f^{-1}(C \cap D) \subseteq f^{-1}(C) \cap f^{-1}(D).
\end{equation}
Now let $x \in f^{-1}(C) \cap f^{-1}(D)$. Then
\begin{align}
	&\implies f(x) \in C \textrm{ and } f(x) \in D \\
	&\implies f(x) \in C \cap D \\
	&\implies x \in f^{-1}(C \cap D).
\end{align}
Thus,
\begin{equation}\label{1a-2}
	 f^{-1}(C) \cap f^{-1}(D) \subseteq f^{-1}(C \cap D).
\end{equation}
Therefore by equations \eqref{1a-1} and \eqref{1a-2}, $f^{-1}(C \cap D) = f^{-1}(C) \cap f^{-1}(D)$.
\end{proof}

\subsection*{(b)}
\textbf{Claim}: If $E \subset \R$ is countable, then the complement $\R\backslash E$ is always uncountable.
\begin{proof}
	(By contradiction). Suppose $E^c$ is countable. Then $E \cup E^c$ is countable as well, since it is the union of two countable sets. But $E \cup E^c = \R$, which is uncountable. $(\Rightarrow\Leftarrow)$.
\end{proof}

\subsection*{(c)}
By contrast, if $E \subset \R$ is uncountable, then the complement $\R \backslash E$ is not always countable. Take for instance, $E = [0 , 1]$, which is uncountable. Then \\$E^c = (-\infty, 0) \cup (0, \infty)$, which is also uncountable.
%%%%%%%%%%%%%%%%%%%%%%%%%%%%%%%%%%%%%%%%%%%%%%%%%%%%%%%%%%%%%%%%%%%%%%%%%%%%%%%%%%%%%%%%%%%%%%%%%%%%%%%%%%%%%%%%%%%%%%%%%%%%%%%%%%%%%%%%%%%%
\section*{Problem 2}
\subsection*{(a)}
A set $U \subset \R$ is \textit{not open} if for every $\epsilon > 0$, $\exists x \in U$ such that \\$(x-\epsilon, x+\epsilon) \not\subset \R$.
\end{document}
	