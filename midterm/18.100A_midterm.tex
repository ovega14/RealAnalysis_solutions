\documentclass{article}
\usepackage[utf8]{inputenc}
\usepackage[english]{babel}
\usepackage[]{amsthm} %lets us use \begin{proof}
\usepackage[]{amssymb} %gives us the character \varnothing
\usepackage[]{amsmath}
\usepackage[parfill]{parskip} %avoid indent when skipping lines
\usepackage[toc,page]{appendix}
\usepackage{mathtools}
\usepackage{hyperref}
\hypersetup{
	colorlinks=true,
	linkcolor=blue,
	filecolor=magenta,      
	urlcolor=cyan,
	pdftitle={18.100A-midterm},
	pdfpagemode=FullScreen,
}
%\urlstyle{same}
\newcommand{\R}{\mathbb{R}} %the real numbers
\newcommand{\N}{\mathbb{N}} %the natural numbers
\newcommand{\M}{\mathcal{M}} %set of all Lebesgue-measurable sets
\newcommand{\Q}{\mathbb{Q}} % the rational numbers

\title{18.100A Midterm}
\author{Octavio Vega}
\date\today

\begin{document}
\maketitle

\section*{Problem 1}
\subsection*{(a)}
\begin{proof}
Let $x \in f^{-1}(C \cap D)$. Then 
\begin{align}
	&\implies f(x) \in C \cap D \\
	&\implies f(x) \in C \textrm{ and } f(x) \in D \\
	&\implies x \in f^{-1}(C) \textrm{ and } x \in f^{-1}(D) \\
	&\implies x \in f^{-1}(C) \cap f^{-1}(D).
\end{align}
Thus, 
\begin{equation}\label{1a-1}
	f^{-1}(C \cap D) \subseteq f^{-1}(C) \cap f^{-1}(D).
\end{equation}
Now let $x \in f^{-1}(C) \cap f^{-1}(D)$. Then
\begin{align}
	&\implies f(x) \in C \textrm{ and } f(x) \in D \\
	&\implies f(x) \in C \cap D \\
	&\implies x \in f^{-1}(C \cap D).
\end{align}
Thus,
\begin{equation}\label{1a-2}
	 f^{-1}(C) \cap f^{-1}(D) \subseteq f^{-1}(C \cap D).
\end{equation}
Therefore by equations \eqref{1a-1} and \eqref{1a-2}, $f^{-1}(C \cap D) = f^{-1}(C) \cap f^{-1}(D)$.
\end{proof}

\subsection*{(b)}
\textbf{Claim}: If $E \subset \R$ is countable, then the complement $\R\backslash E$ is always uncountable.
\begin{proof}
	(By contradiction). Suppose $E^c$ is countable. Then $E \cup E^c$ is countable as well, since it is the union of two countable sets. But $E \cup E^c = \R$, which is uncountable. $(\Rightarrow\Leftarrow)$.
\end{proof}

\subsection*{(c)}
By contrast, if $E \subset \R$ is uncountable, then the complement $\R \backslash E$ is not always countable. Take for instance, $E = [0 , 1]$, which is uncountable. Then \\$E^c = (-\infty, 0) \cup (1, \infty)$, which is also uncountable.
%%%%%%%%%%%%%%%%%%%%%%%%%%%%%%%%%%%%%%%%%%%%%%%%%%%%%%%%%%%%%%%%%%%%%%%%%%%%%%%%%%%%%%%%%%%%%%%%%%%%%%%%%%%%%%%%%%%%%%%%%%%%%%%%%%%%%%%%%%%%
\section*{Problem 2}
\subsection*{(a)}
A set $U \subset \R$ is \textit{not open} if for every $\epsilon > 0$, $\exists x \in U$ such that \\$(x-\epsilon, x+\epsilon) \not\subset \R$.

\subsection*{(b)}
\begin{proof}
	Suppose $U$ is not open. Let $\epsilon = \frac{1}{n}$. Then for every $n \in \N$, $\exists x \in U$ such that $(x-\frac{1}{n}, x + \frac{1}{n}) \not \subset U$. Equivalently, for each $n \in \N$ $\exists x_n \in U^c$ such that
	\begin{equation}
		x - \frac{1}{n} < x_n < x + \frac{1}{n}.
	\end{equation}
	Then we have
	\begin{equation}
		0 < |x_n - x| < \frac{1}{n},
	\end{equation}
	and taking the limit on all sides gives
	\begin{equation}
		0 < \lim_{n \to \infty} |x_n - x| < \lim_{n \to \infty} \frac{1}{n}.
	\end{equation}
	Thus, by the squeeze theorem, $\lim_{n \to \infty} x_n = x$, as desired.
\end{proof}

\subsection*{(c)}
\begin{proof}
	(By contradiction). To show that $F$ is closed, we must show that $F^c$ is open. Suppose, toward a contradiction, that $F^c$ is not open. Then by part \textbf{(b)}, $\exists x \in F^c$ and a sequence $\{x_n\}_n$ of elements of $F$ such that $\lim_{n\to\infty} x_n = x$. But by assumption, every convergent sequence of elements of $F$ has a limit in $F$, i.e. we assumed originally that $x \in F$ $(\Rightarrow\Leftarrow)$. Thus, $F^c$ must be open, so $F$ is closed.
\end{proof}
%%%%%%%%%%%%%%%%%%%%%%%%%%%%%%%%%%%%%%%%%%%%%%%%%%%%%%%%%%%%%%%%%%%%%%%%%%%%%%%%%%%%%%%%%%%%%%%%%%%%%%%%%%%%%%%%%%%%%%%%%%%%%%%%%%%%%%%%%%%%
\section*{Problem 3}
\subsection*{(a)}
\begin{proof}
	Let $\epsilon > 0$. Choose $N = \frac{1}{\sqrt{\epsilon}}$. Then $\forall n \geq N$, we have
	\begin{align}
		\left|\frac{10n^2}{n^2 + 16n + 1} - 10\right| &= \left|\frac{-160n - 10}{n^2 + 16n + 1}\right| \\
		&= \left|\frac{160n + 10}{n^2 + 16n + 1}\right| \\
		&< \frac{1}{n^2 + 16n + 1} \\
		&< \frac{1}{n^2} \\
		&< \epsilon.
	\end{align}
	Therefore, $\lim_{n \to \infty} \left|\frac{10n^2}{n^2 + 16n + 1}\right| = 10$.
\end{proof}

\subsection*{(b)}
(i) Let $x_n = \frac{(-1)^n}{n}$. Then $\lim_{n \to \infty} x_n = 0$. But $x_1 = -1 < \frac{1}{2} = x_2$, whereas $x_2 = \frac{1}{2} > -\frac{1}{3} = x_3$. 

Therefore $\{x_n\}_n$ converges to $0$, but is not monotonic.

(ii) Let 
\begin{equation}
	x_n = \begin{cases}
		0, \quad n \textrm{ even} \\
		n, \quad n \textrm{ odd}.
	\end{cases}
\end{equation}
Then $\{x_n\}_n$ is clearly unbounded.

Consider the subsequence $\{x_{n_k}\}_k$ defined by $x_{n_k} = x_{2k}$ for each $k \in \N$. Then $\forall k$, $x_{n_k} = 0$.

Therefore, $\{x_n\}_n$ is unbounded, but $\{x_{n_k}\}_k$ converges to $0$. 
%%%%%%%%%%%%%%%%%%%%%%%%%%%%%%%%%%%%%%%%%%%%%%%%%%%%%%%%%%%%%%%%%%%%%%%%%%%%%%%%%%%%%%%%%%%%%%%%%%%%%%%%%%%%%%%%%%%%%%%%%%%%%%%%%%%%%%%%%%%%
\section*{Problem 4}
\subsection*{(a)}
(i) 
\begin{proof}
	Suppose $\{x_n\}_n$ and $\{y_n\}_n$ are bounded. Then $\exists B_0 > 0$ and $B_1 > 0$ such that $\forall n \in \N$, $|x_n| \leq B_0$ and $|y_n| \leq B_1$.
	
	Let $B = B_0 + B_1$. Then by the triangle inequality, we have
	\begin{equation}
		|x_n + y_n| \leq |x_n| + |y_n| \leq B_0 + B_1 = B.
	\end{equation}
	Thus, $\forall n \in \N$, $|x_n + y_n| \leq B$.
	
	Therefore, $\{x_n + y_n\}_n$ is bounded.
\end{proof}

(ii) 
\begin{proof}
	Let $a_n = \sup\{x_k \;|\; k\geq n\}$, $b_n = \sup\{y_k \;|\; k\geq n\}$, and \\$c_n = \sup\{x_k + y_k \;|\; k\geq n\}$.
	
	Then $\forall n \in \N$, $a_n \geq x_n$ and $b_n \geq y_n$, so
	\begin{equation}
		\implies x_n + y_n \leq a_n + b_n.
	\end{equation}
	But $a_n + b_n$ is also an upper bound for $\{x_n + y_n\}_n$, so $c_n \leq a_n + b_n$. Then
	\begin{equation}
		\lim_{n \to \infty} c_n \leq \lim_{n \to \infty} a_n + \lim_{n \to \infty} b_n.
	\end{equation}
	Therefore, $\limsup(x_n + y_n) \leq \limsup x_n + \limsup y_n$.
\end{proof}

\end{document}
	