\documentclass{article}
\usepackage[utf8]{inputenc}
\usepackage[english]{babel}
\usepackage[]{amsthm} %lets us use \begin{proof}
\usepackage[]{amssymb} %gives us the character \varnothing
\usepackage[]{amsmath}
\usepackage[parfill]{parskip} %avoid indent when skipping lines
\usepackage[toc,page]{appendix}
\usepackage{mathtools}
\usepackage{hyperref}
\hypersetup{
	colorlinks=true,
	linkcolor=blue,
	filecolor=magenta,      
	urlcolor=cyan,
	pdftitle={18.100A-ps10},
	pdfpagemode=FullScreen,
}
%\urlstyle{same}
\newcommand{\R}{\mathbb{R}} %the real numbers
\newcommand{\N}{\mathbb{N}} %the natural numbers
\newcommand{\M}{\mathcal{M}} %set of all Lebesgue-measurable sets
\newcommand{\Q}{\mathbb{Q}} % the rational numbers

\title{18.100A Assignment 10}
\author{Octavio Vega}
\date\today

\begin{document}
\maketitle
	
\section*{Problem 1}
\subsection*{(a)}
\begin{proof}
	Suppose $\exists C \geq 0$ such that $\forall x, y \in I$,
	\begin{equation}
		|f(x) - f(y)| \leq C |x - y|^\alpha.
	\end{equation}
	Let $\epsilon > 0$. Choose $\delta = \left(\frac{\epsilon}{C}\right)^\frac{1}{\alpha}$. Then if $|x - y| < \delta$, we get
	\begin{align}
		|f(x) - f(y)| & \leq C |x - y|^\alpha \\
		& < C \delta^\alpha \\
		&= C \frac{\epsilon}{C} \\
		&= \epsilon.
	\end{align}
	Therefore $f$ is uniformly continuous on $I$.
\end{proof}

\subsection*{(b)}
\begin{proof}
	Suppose $\exists C \geq 0$ such that $\forall x, y \in I$, $|f(x) - f(y)| \leq C |x - y|^\alpha$.
	
	Since $\alpha > 1$, then $\alpha = 1 + r$ for some $0 < r$, we have
	\begin{align}
		& \implies 0 \leq |f(x) - f(y)| \leq C |x - y|^{1 + r} \\
		& \implies 0 \leq \frac{|f(x) - f(y)|}{|x - y|} \leq C |x - y|^r \\
		& \implies \lim\limits_{x \to y} 0 \leq \lim\limits_{x \to y} \frac{|f(x) - f(y)|}{|x - y|} \leq C \lim\limits_{x \to y}|x - y|^r \\
		& \implies 0 \leq \lim\limits_{x \to y} \frac{|f(x) - f(y)|}{|x - y|} \leq 0.
	\end{align} 
	Then by the squeeze theorem, $\lim \limits_{x \to y} \frac{|f(x) - f(y)|}{|x - y|} = 0$. Thus $\forall y \in I$, $f'(y) = 0$.
	
	Therefore $f$ is constant. 
\end{proof}
%%%%%%%%%%%%%%%%%%%%%%%%%%%%%%%%%%%%%%%%%%%%%%%%%%%%%%%%%%%%%%%%%%%%%%%%%%%%%%%%%%%%%%%%%%%%%%%%%%%%%%%%%%%%%%%%%%%%%%%%%%%%%%%%%%%%%%%%%%%%%%%
\section*{Problem 2}
\begin{proof}
	We compute:
	\begin{align}
		L &= \lim \limits_{x \to c} \frac{h(x) - h(c)}{x - c} \\
		&= \lim\limits_{x \to c} \frac{f(x)g(x) - f(c)g(c)}{x - c} \\
		&= \lim\limits_{x \to c} \frac{f(x)g(x) - f(x)g(c) + f(x)g(c) - f(c)g(c)}{x - c} \\
		&= \lim\limits_{x \to c} \left[f(x) \left(\frac{g(x) - g(c)}{x - c}\right)\right] + g(x) \lim\limits_{x \to c}\left(\frac{f(x) - f(c)}{x - c}\right) \\ 
	\end{align}
	Since $f$ is continuous at $c$, and both $f$ and $g$ are differentiable at $c$, this gives us
	\begin{equation}
		L = f(c)g'(c) + g(c)f'(c),
	\end{equation}
	which exists.
	
	Therefore $f(x)g(x)$ is differentiable at $c$.
\end{proof}
%%%%%%%%%%%%%%%%%%%%%%%%%%%%%%%%%%%%%%%%%%%%%%%%%%%%%%%%%%%%%%%%%%%%%%%%%%%%%%%%%%%%%%%%%%%%%%%%%%%%%%%%%%%%%%%%%%%%%%%%%%%%%%%%%%%%%%%%%%%%%%%

\end{document}