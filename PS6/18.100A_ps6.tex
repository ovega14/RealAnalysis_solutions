\documentclass{article}
\usepackage[utf8]{inputenc}
\usepackage[english]{babel}
\usepackage[]{amsthm} %lets us use \begin{proof}
\usepackage[]{amssymb} %gives us the character \varnothing
\usepackage[]{amsmath}
\usepackage[parfill]{parskip} %avoid indent when skipping lines
\usepackage{mathtools}
\usepackage{hyperref}
\hypersetup{
	colorlinks=true,
	linkcolor=blue,
	filecolor=magenta,      
	urlcolor=cyan,
	pdftitle={18.100A-ps6},
	pdfpagemode=FullScreen,
}
%\urlstyle{same}

\title{18.100A Assignment 6}
\author{Octavio Vega}
\date\today

\begin{document}
\maketitle
	
\section*{Problem 1}
\subsection*{(a)}
\begin{align}
	\sum_{n=1}^{\infty} \frac{3}{9n+1} &= 3\sum_{n=1}^{\infty} \frac{1}{9n+1} \\
	&= \frac{1}{3}\sum_{n=1}^{\infty}\frac{1}{n+\frac{1}{9}} \\
	&= \frac{1}{3}\sum_{n=2}^{\infty}\frac{1}{(n-1)+\frac{1}{9}} \\
	&= \frac{1}{3}\sum_{n=2}^{\infty} \frac{1}{n-\frac{8}{9}} \\
	&> \sum_{n=2}^{\infty} \frac{1}{n}.
\end{align}
But the Harmonic series, $\sum_n \frac{1}{n}$, diverges.

Therefore, we conclude by comparison that the series $\sum_n \frac{3}{9n+1}$ diverges.

\subsection*{(b)}
\begin{equation}
	\sum_{n=1}^{\infty}\frac{1}{2n-1} = \frac{1}{2}\sum_{n=1}^{\infty}\frac{1}{n-\frac{1}{2}} > \frac{1}{2}\sum_{n=1}^{\infty}\frac{1}{n}.
\end{equation}
Therefore, by comparison, $\sum_n \frac{1}{2n-1}$ diverges.

\subsection*{(c)}
\begin{equation}
	\sum_{n=1}^{\infty} \frac{(-1)^n}{n^2} = \sum_{k=1}^{\infty}\frac{1}{(2k)^2} - \sum_{k=1}^{\infty}\frac{1}{(2k+1)^2},
\end{equation}
which is the difference of two convergent series. 

Therefore, $\sum_n \frac{(-1)^n}{n^2}$ converges.

\subsection*{(d)}
We can express the series $\sum_n \frac{1}{n(n+1)}$ as a telescoping sum:
\begin{align}
	\sum_{n=1}^{\infty}\frac{1}{n(n+1)} &= \sum_{n=1}^{\infty} \frac{1}{n} - \sum_{n=1}^{\infty}\frac{1}{n+1} \\
	&= \left(1-\frac{1}{2}\right) + \left(\frac{1}{2} - \frac{1}{3}\right) + \left(\frac{1}{3} - \frac{1}{4}\right) + \cdots \\
	&= 1.
\end{align}
Therefore, $\sum_n \frac{1}{n(n+1)}$ converges to $1$.

\subsection*{(e)}
We note that $\forall n \in \mathbb{N}$, $e^{n^2} \geq n^3$. Then we have
\begin{equation}
	\sum_{n=1}^{\infty}\frac{n}{e^{n^2}} \leq \sum_{n=1}^{\infty} \frac{n}{n^3} = \sum_{n=1}^{\infty}\frac{1}{n^2}, 
\end{equation}
which converges. 

Therefore, $\sum_n ne^{-n^2}$ converges.
%%%%%%%%%%%%%%%%%%%%%%%%%%%%%%%%%%%%%%%%%%%%%%%%%%%%%%%%%%%%%%%%%%%%%%%%%%%%%%%%%%%%%%%%%%%%%%%%%%%%%%%%%%%%%%%%%%%%%%%%%%%%%%%%%%%%%%%%%%%%
\section*{Problem 2}
\subsection*{(a)}
\begin{proof}
	Suppose $\sum_n x_n$ converges. 
	
	Then the sequence of partial sums $\{s_m\}_m = \{x_1 + \cdots x_m\}_{m=1}^{\infty}$ also converges. 
	
	We construct two subsequences of $\{s_m\}_m$, defined via
	\begin{equation}
		\{s_{m_{2k}}\}_k = \{x_2, x_2 + x_4, x_2 + x_4 + x_6, \cdots\} = \{x_2 + \cdots x_{2k}\}_k,
	\end{equation}
	and
	\begin{equation}
		\{s_{m_{2k+1}}\}_k = \{x_1, x_1 + x_3, x_1 + x_3 + x_7, \cdots\} = \{x_1 + \cdots x_{2k+1}\}_k.
	\end{equation}
	Note that for each $m\in\mathbb{N}$, if $m$ is even then $m=2k$ for some $k\in\mathbb{N}$. Then
	\begin{equation}
		s_m = s_{m_{2k-1}} + s_{m_{2k}}.
	\end{equation}
	Likewise if $m$ is odd, then
	\begin{equation}
		s_m = s_{m_{2k}} + s_{m_{2k+1}}.
	\end{equation}
	In both cases, the partial sum $s_m$ must converge because the series $\sum_n x_n$ converges. Thus, the sum of the two partial sums must also converge, i.e. $\{s_{m_{2k}}\}_k$ and $\{s_{m_{2k+1}}\}_k$ both converge. 
	
	Then both series $\sum_n x_{2n}$ and $\sum_n x_{2n+1}$ must converge, and so does their sum:
	\begin{equation}
		\sum_n x_{2n} + \sum_n x_{2n+1} < \infty.
	\end{equation}
	Therefore, $\sum_n \left(x_{2n} + x_{2n+1}\right)$ converges.
\end{proof}

\subsection*{(b)}
Consider the following counterexample to the converse of the theorem proven in problem \textbf{(a)}:

Let $x_n = (-1)^n$. Then 
\begin{align}
	\sum_{n=1}^{\infty}\left(x_{2n} +x_{2n+1}\right) &= \sum_{n=1}^{\infty}\left[(-1)^{2n} + (-1)^{2n+1}\right] \\
	&= \sum_{n=1}^{\infty}\left(1^n - 1^n \right) \\
	&= 0.
\end{align}
But $\sum_n (-1)^n$ does not converge. 

Therefore, the converse to \textbf{(a)} does not always hold.
%%%%%%%%%%%%%%%%%%%%%%%%%%%%%%%%%%%%%%%%%%%%%%%%%%%%%%%%%%%%%%%%%%%%%%%%%%%%%%%%%%%%%%%%%%%%%%%%%%%%%%%%%%%%%%%%%%%%%%%%%%%%%%%%%%%%%%%%%%%%
\section*{Problem 3}
\begin{proof}
	Suppose $\sum_n x_n$ converges absolutely.
	
	Since the series $\sum_n |x_n|$ converges, then $\sum_n x_n$ converges by comparison, because $x_n \leq |x_n|$.
	
	Let $N \in \mathbb{N}$. Then the triangle inequality gives
	\begin{equation}
		|x_1 + x_2 + \cdots + x_N| \leq |x_1| + |x_2| + \cdots + |x_N| = \sum_{n=1}^{N}|x_n|.
	\end{equation}
	Because $\sum_n |x_n|$ converges, we can now take $N \to \infty$, which gives
	\begin{equation}
		\left|\sum_{n=1}^{\infty} x_n \right| \leq \sum_{n=1}^{\infty} |x_n|,
	\end{equation}
	as desired.
\end{proof}
%%%%%%%%%%%%%%%%%%%%%%%%%%%%%%%%%%%%%%%%%%%%%%%%%%%%%%%%%%%%%%%%%%%%%%%%%%%%%%%%%%%%%%%%%%%%%%%%%%%%%%%%%%%%%%%%%%%%%%%%%%%%%%%%%%%%%%%%%%%%
\end{document}