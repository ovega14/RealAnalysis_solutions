\documentclass{article}
\usepackage[utf8]{inputenc}
\usepackage[english]{babel}
\usepackage[]{amsthm} %lets us use \begin{proof}
\usepackage[]{amssymb} %gives us the character \varnothing
\usepackage[]{amsmath}
\usepackage[parfill]{parskip} %avoid indent when skipping lines

\title{18.100A Assignment 3}
\author{Octavio Vega}
\date\today

\begin{document}
\maketitle
	
\section*{Problem 1}
\begin{proof}
	Let $x,y \in \mathbb{R}$. By the density of $\mathbb{Q}$, we have that $\exists r \in \mathbb{Q}$ such that $x<r<y$. 
	
	Then $x+\sqrt{2} < y + \sqrt{2}$. Then $\exists r \in \mathbb{Q}$ such that
	\begin{align}
		x + \sqrt{2} < r < y + \sqrt{2} \\
		\implies x < r - \sqrt{2} < y.
	\end{align}
	But since $r \in \mathbb{Q}$ and $\sqrt{2} \notin \mathbb{Q}$, then the number $i := r-\sqrt{2} \notin \mathbb{Q}$.
	
	So $x<i<y$ with $i \in \mathbb{R} \backslash \mathbb{Q}$, as desired.
\end{proof}

\section*{Problem 2}
\begin{proof}
	Define the function $f: E\longrightarrow \wp(\mathbb{N})$ such that if $x=0.d_{-1}d_{-2}...$, then 
	\begin{equation}
		f(x) = \{j \in \mathbb{N} \;|\; d_{-j} = 2\}.
	\end{equation}
	We want to show that $f$ is a bijection. First, we show that $f$ is injective.
	
	Let $x_1=0.d_{-1}^{(1)}d_{-2}^{(1)}...$ and $x_2=0.d_{-1}^{(2)}d_{-2}^{(2)}...$ for $x_1, x_2 \in E$. Suppose $f(x_1) = f(x_2)$. Then
	\begin{equation}
		\{j\in\mathbb{N} \;|\; d_{-j}^{(1)}=2\} = \{k\in\mathbb{N} \;|\; d_{-k}^{(2)}=2\}.
	\end{equation}
	Since each digit $d_{-j}\in\{1, 2\}$, then the sets of digits must be the same:
	\begin{equation}
		\{d_{-j}^{(1)} \;|\; j\in\mathbb{N}\} = \{d_{-k}^{(2)} \;|\; k\in\mathbb{N}\}.
	\end{equation}
	But by the theorem from class, we know that for every set of digits $\exists ! x\in[0,1]$ such that $x=0.d_{-1}d_{-2}...$. So if all of the digits are the same, then the numbers must be the same, i.e.
	\begin{equation}
		f(x_1) = f(x_2) \implies x_1 = x_2.
	\end{equation}
	Thus $f$ is injective.
	
	Next, we show that $f$ is surjective. 
	
	Let $S\in\wp(\mathbb{N})$ with
	\begin{equation}
		S:=\{j\in\mathbb{N} \;|\; d_{-j}=2\}.
	\end{equation}
	 Since this is a set of digits, then by the theorem from class $\exists x\in [0,1]$ such that $x=0.d_{-1}d_{-2}...$; i.e. for any $S\in\wp(\mathbb{N})$, $\exists x\in E$ such that $f(x) = S$. 
	 
	 Hence, $f$ is also surjective, which means that it is bijective.
	 
	 Therefore we conclude that $|E|=|\wp(\mathbb{N})|$.
\end{proof}
\end{document}