\documentclass{article}
\usepackage[utf8]{inputenc}
\usepackage[english]{babel}
\usepackage[]{amsthm} %lets us use \begin{proof}
\usepackage[]{amssymb} %gives us the character \varnothing
\usepackage[]{amsmath}
\usepackage[parfill]{parskip} %avoid indent when skipping lines

\title{18.100A Assignment 3}
\author{Octavio Vega}
\date\today

\begin{document}
\maketitle
	
\section*{Problem 1}
\begin{proof}
	Let $x,y \in \mathbb{R}$ with $x<y$. By the density of $\mathbb{Q}$, we have that $\exists r \in \mathbb{Q}$ such that $x<r<y$. 
	
	Then $x+\sqrt{2} < y + \sqrt{2}$. Then $\exists r \in \mathbb{Q}$ such that
	\begin{align}
		x + \sqrt{2} < r < y + \sqrt{2} \\
		\implies x < r - \sqrt{2} < y.
	\end{align}
	But since $r \in \mathbb{Q}$ and $\sqrt{2} \notin \mathbb{Q}$, then the number $i := r-\sqrt{2} \notin \mathbb{Q}$.
	
	So $x<i<y$ with $i \in \mathbb{R} \backslash \mathbb{Q}$, as desired.
\end{proof}
%%%%%%%%%%%%%%%%%%%%%%%%%%%%%%%%%%%%%%%%%%%%%%%%%%%%%%%%%%%%%%%%%%%%%%%%%%%%%%%%%%%%%%%%%%%%%%%%%%%%%%%%%%%%%%%%%%%%%%%%%%%%%%%%%%%%%%%%%
\section*{Problem 2}
\begin{proof}
	Define the function $f: E\rightarrow \wp(\mathbb{N})$ such that if $x=0.d_{-1}d_{-2}...$, then 
	\begin{equation}
		f(x) = \{j \in \mathbb{N} \;|\; d_{-j} = 2\}.
	\end{equation}
	We want to show that $f$ is a bijection. First, we show that $f$ is injective.
	
	Let $x_1=0.d_{-1}^{(1)}d_{-2}^{(1)}...$ and $x_2=0.d_{-1}^{(2)}d_{-2}^{(2)}...$ for $x_1, x_2 \in E$. Suppose $f(x_1) = f(x_2)$. Then
	\begin{equation}
		\{j\in\mathbb{N} \;|\; d_{-j}^{(1)}=2\} = \{k\in\mathbb{N} \;|\; d_{-k}^{(2)}=2\}.
	\end{equation}
	Since each digit $d_{-j}\in\{1, 2\}$, then the sets of digits must be the same:
	\begin{equation}
		\{d_{-j}^{(1)} \;|\; j\in\mathbb{N}\} = \{d_{-k}^{(2)} \;|\; k\in\mathbb{N}\}.
	\end{equation}
	But by the theorem from class, we know that for every set of digits $\exists ! x\in[0,1]$ such that $x=0.d_{-1}d_{-2}...$. So if all of the digits are the same, then the numbers must be the same, i.e.
	\begin{equation}
		f(x_1) = f(x_2) \implies x_1 = x_2.
	\end{equation}
	Thus $f$ is injective.
	
	Next, we show that $f$ is surjective. 
	
	Let $S\in\wp(\mathbb{N})$ with
	\begin{equation}
		S:=\{j\in\mathbb{N} \;|\; d_{-j}=2\}.
	\end{equation}
	 Since this corresponds to the indices for a set of digits, then by the theorem from class $\exists x\in [0,1]$ such that $x=0.d_{-1}d_{-2}...$; i.e. for any $S\in\wp(\mathbb{N})$, $\exists x\in E$ such that $f(x) = S$. 
	 
	 Hence, $f$ is also surjective, which means that it is bijective.
	 
	 Therefore we conclude that $|E|=|\wp(\mathbb{N})|$.
\end{proof}
%%%%%%%%%%%%%%%%%%%%%%%%%%%%%%%%%%%%%%%%%%%%%%%%%%%%%%%%%%%%%%%%%%%%%%%%%%%%%%%%%%%%%%%%%%%%%%%%%%%%%%%%%%%%%%%%%%%%%%%%%%%%%%%%%%%%%%
\section*{Problem 3}
\subsection*{(a)}
\begin{proof}
	We want to show that there exists a bijection $h: A\cup B\rightarrow \mathbb{N}$. 
	
	Recall that we can construct the sets of even and odd natural numbers as follows:
	\begin{equation}
		\mathcal{O} := \{2n+1 \;|\; n\in \mathbb{N}\} \textrm{, and } \mathcal{E}:=\{2n \;|\; n\in\mathbb{N}\}.
	\end{equation}
	We also know that $|\mathcal{O}| = |\mathcal{E}| = |\mathbb{N}|$ because the functions defined via $f_e(n)=2n$ and $f_o(n)=2n+1$, mapping $\mathcal{E}$ to $\mathbb{N}$ and $\mathcal{O}$ to $\mathbb{N}$, respectively, are both bijective. Finally, we note that $\mathcal{O}\cup\mathcal{E} = \mathbb{N}$.
	
	Since $A$ and $B$ are both countably infinite, i.e. $|A|=|B|=|\mathbb{N}|$, then $\exists$ bijections $f,g$ such that 
	\begin{equation}
		f: a\rightarrow \mathbb{N} \textrm{, and } g:B\rightarrow\mathbb{N}.
	\end{equation}
	Then $\forall a \in A$, $2f(a)$ is even, and $\forall b \in B$, $2g(b)+1$ is odd. So we define the even and odd sets
	\begin{equation}
		\mathcal{E} := \{2f(a) \;|\; a \in A\} \textrm{, and } \mathcal{O} := \{2g(b)+1 \;|\; b \in B\}.
	\end{equation}
	Then we construct the function $h$ defined via
	\begin{equation}
		h(x)=
		\begin{cases}
			2f(x) & \text{if } x \in A\\
			2g(x)+1 & \text{if } x \in B.
		\end{cases}
	\end{equation}
	First, we show that $h$ is injective.
	
	\underline{Case 1}: $x,y\in A$. Then 
	\begin{align}
		&\quad h(x) = h(y) \\
		&\implies 2f(x) = 2f(y) \\
		&\implies f(x) = f(y) \\
		&\implies x = y,
	\end{align}
	Since $f$ is injective.
	
	\underline{Case 2}: $x,y\in B$. Then
	\begin{align}
		&\quad h(x) = h(y) \\
		&\implies 2g(x) + 1 = 2g(y) + 1 \\
		&\implies g(x) = g(y) \\
		&\implies x = y,
	\end{align}
	since $g$ is injective. 
	
	\underline{Case 3}: $x\in A$, $y\in B$ WOLOG, and $h(x)=h(y)$. This case is vacuous because $\mathcal{E}$ and $\mathcal{O}$ are disjoint, whereas $2f(x)\in\mathcal{E}$ while $2g(x) + 1 \in\mathcal{O}$.
	
	Thus, according to cases 1 and 2, $h$ is injective.
	
	Now we show that $h$ is surjective. Let $n\in\mathbb{N}$.
	
	\underline{Case 1}: $n$ is even. Then by surjectivity of $f$, $\exists a \in A$ such that $f(a)=\frac{n}{2}$, i.e. $h(a)=n$.
	
	\underline{Case 2}: $n$ is odd. Then by surjectivity of $g$, $\exists b \in B$ such that $g(b)=\frac{n-1}{2}$, i.e. $h(b)=n$.
	
	Thus, $h$ is surjective as well.
	
	Therefore we conclude that $h$ is a bijection, hence $|A\cup B| = |\mathbb{N}|$.
\end{proof}

\subsection*{(b)}
\begin{proof}
	(By contradiction). Suppose instead that $\mathbb{R}\backslash \mathbb{Q}$ is countable.  We write:
	\begin{align}
		\mathbb{R} &= \mathbb{R}\backslash \mathbb{Q} \cup \mathbb{Q} \\
		\implies |\mathbb{R}| &= |\mathbb{R}\backslash \mathbb{Q} \cup \mathbb{Q}|.
	\end{align}
	We know that $\mathbb{Q}$ is countably infinite, so $|\mathbb{Q}| = |\mathbb{N}|$. We also assumed that $\mathbb{R}\backslash\mathbb{Q}$ is countable, so $|\mathbb{R}\backslash \mathbb{Q}|= |\mathbb{N}|$. Then by part \textbf{(a)}, we have that $|\mathbb{R}| = |\mathbb{N}|$, but the reals are uncountable, so contradiction $(\Rightarrow \Leftarrow)$.
	
	Hence $\mathbb{R}\backslash\mathbb{Q}$ must be uncountable.
\end{proof}

\end{document}