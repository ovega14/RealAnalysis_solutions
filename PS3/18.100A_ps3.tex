\documentclass{article}
\usepackage[utf8]{inputenc}
\usepackage[english]{babel}
\usepackage[]{amsthm} %lets us use \begin{proof}
\usepackage[]{amssymb} %gives us the character \varnothing
\usepackage[]{amsmath}
\usepackage[parfill]{parskip} %avoid indent when skipping lines

\title{18.100A Assignment 3}
\author{Octavio Vega}
\date\today

\begin{document}
\maketitle
	
\section*{Problem 1}
\begin{proof}
	Let $x,y \in \mathbb{R}$ with $x<y$. By the density of $\mathbb{Q}$, we have that $\exists r \in \mathbb{Q}$ such that $x<r<y$. 
	
	Then $x+\sqrt{2} < y + \sqrt{2}$. Then $\exists r \in \mathbb{Q}$ such that
	\begin{align}
		x + \sqrt{2} < r < y + \sqrt{2} \\
		\implies x < r - \sqrt{2} < y.
	\end{align}
	But since $r \in \mathbb{Q}$ and $\sqrt{2} \notin \mathbb{Q}$, then the number $i := r-\sqrt{2} \notin \mathbb{Q}$.
	
	So $x<i<y$ with $i \in \mathbb{R} \backslash \mathbb{Q}$, as desired.
\end{proof}
%%%%%%%%%%%%%%%%%%%%%%%%%%%%%%%%%%%%%%%%%%%%%%%%%%%%%%%%%%%%%%%%%%%%%%%%%%%%%%%%%%%%%%%%%%%%%%%%%%%%%%%%%%%%%%%%%%%%%%%%%%%%%%%%%%%%%%%%%
\section*{Problem 2}
\begin{proof}
	Define the function $f: E\rightarrow \wp(\mathbb{N})$ such that if $x=0.d_{-1}d_{-2}...$, then 
	\begin{equation}
		f(x) = \{j \in \mathbb{N} \;|\; d_{-j} = 2\}.
	\end{equation}
	We want to show that $f$ is a bijection. First, we show that $f$ is injective.
	
	Let $x_1=0.d_{-1}^{(1)}d_{-2}^{(1)}...$ and $x_2=0.d_{-1}^{(2)}d_{-2}^{(2)}...$ for $x_1, x_2 \in E$. Suppose $f(x_1) = f(x_2)$. Then
	\begin{equation}
		\{j\in\mathbb{N} \;|\; d_{-j}^{(1)}=2\} = \{k\in\mathbb{N} \;|\; d_{-k}^{(2)}=2\}.
	\end{equation}
	Since each digit $d_{-j}\in\{1, 2\}$, then the sets of digits must be the same:
	\begin{equation}
		\{d_{-j}^{(1)} \;|\; j\in\mathbb{N}\} = \{d_{-k}^{(2)} \;|\; k\in\mathbb{N}\}.
	\end{equation}
	But by the theorem from class, we know that for every set of digits $\exists ! x\in[0,1]$ such that $x=0.d_{-1}d_{-2}...$. So if all of the digits are the same, then the numbers must be the same, i.e.
	\begin{equation}
		f(x_1) = f(x_2) \implies x_1 = x_2.
	\end{equation}
	Thus $f$ is injective.
	
	Next, we show that $f$ is surjective. 
	
	Let $S\in\wp(\mathbb{N})$ with
	\begin{equation}
		S:=\{j\in\mathbb{N} \;|\; d_{-j}=2\}.
	\end{equation}
	 Since this corresponds to the indices for a set of digits, then by the theorem from class $\exists x\in [0,1]$ such that $x=0.d_{-1}d_{-2}...$; i.e. for any $S\in\wp(\mathbb{N})$, $\exists x\in E$ such that $f(x) = S$. 
	 
	 Hence, $f$ is also surjective, which means that it is bijective.
	 
	 Therefore we conclude that $|E|=|\wp(\mathbb{N})|$.
\end{proof}
%%%%%%%%%%%%%%%%%%%%%%%%%%%%%%%%%%%%%%%%%%%%%%%%%%%%%%%%%%%%%%%%%%%%%%%%%%%%%%%%%%%%%%%%%%%%%%%%%%%%%%%%%%%%%%%%%%%%%%%%%%%%%%%%%%%%%%
\section*{Problem 3}
\subsection*{(a)}
\begin{proof}
	We want to show that there exists a bijection $h: A\cup B\rightarrow \mathbb{N}$. 
	
	Recall that we can construct the sets of even and odd natural numbers as follows:
	\begin{equation}
		\mathcal{O} := \{2n+1 \;|\; n\in \mathbb{N}\} \textrm{, and } \mathcal{E}:=\{2n \;|\; n\in\mathbb{N}\}.
	\end{equation}
	We also know that $|\mathcal{O}| = |\mathcal{E}| = |\mathbb{N}|$ because the functions defined via $f_e(n)=2n$ and $f_o(n)=2n+1$, mapping $\mathcal{E}$ to $\mathbb{N}$ and $\mathcal{O}$ to $\mathbb{N}$, respectively, are both bijective. Finally, we note that $\mathcal{O}\cup\mathcal{E} = \mathbb{N}$.
	
	Since $A$ and $B$ are both countably infinite, i.e. $|A|=|B|=|\mathbb{N}|$, then $\exists$ bijections $f,g$ such that 
	\begin{equation}
		f: a\rightarrow \mathbb{N} \textrm{, and } g:B\rightarrow\mathbb{N}.
	\end{equation}
	Then $\forall a \in A$, $2f(a)$ is even, and $\forall b \in B$, $2g(b)+1$ is odd. So we define the even and odd sets
	\begin{equation}
		\mathcal{E} := \{2f(a) \;|\; a \in A\} \textrm{, and } \mathcal{O} := \{2g(b)+1 \;|\; b \in B\}.
	\end{equation}
	Then we construct the function $h$ defined via
	\begin{equation}
		h(x)=
		\begin{cases}
			2f(x) & \text{if } x \in A\\
			2g(x)+1 & \text{if } x \in B.
		\end{cases}
	\end{equation}
	First, we show that $h$ is injective.
	
	\underline{Case 1}: $x,y\in A$. Then 
	\begin{align}
		&\quad h(x) = h(y) \\
		&\implies 2f(x) = 2f(y) \\
		&\implies f(x) = f(y) \\
		&\implies x = y,
	\end{align}
	Since $f$ is injective.
	
	\underline{Case 2}: $x,y\in B$. Then
	\begin{align}
		&\quad h(x) = h(y) \\
		&\implies 2g(x) + 1 = 2g(y) + 1 \\
		&\implies g(x) = g(y) \\
		&\implies x = y,
	\end{align}
	since $g$ is injective. 
	
	\underline{Case 3}: $x\in A$, $y\in B$ WOLOG, and $h(x)=h(y)$. This case is vacuous because $\mathcal{E}$ and $\mathcal{O}$ are disjoint, whereas $2f(x)\in\mathcal{E}$ while $2g(x) + 1 \in\mathcal{O}$.
	
	Thus, according to cases 1 and 2, $h$ is injective.
	
	Now we show that $h$ is surjective. Let $n\in\mathbb{N}$.
	
	\underline{Case 1}: $n$ is even. Then by surjectivity of $f$, $\exists a \in A$ such that $f(a)=\frac{n}{2}$, i.e. $h(a)=n$.
	
	\underline{Case 2}: $n$ is odd. Then by surjectivity of $g$, $\exists b \in B$ such that $g(b)=\frac{n-1}{2}$, i.e. $h(b)=n$.
	
	Thus, $h$ is surjective as well.
	
	Therefore we conclude that $h$ is a bijection, hence $|A\cup B| = |\mathbb{N}|$.
\end{proof}

\subsection*{(b)}
\begin{proof}
	(By contradiction). Suppose instead that $\mathbb{R}\backslash \mathbb{Q}$ is countable.  We write:
	\begin{align}
		\mathbb{R} &= \mathbb{R}\backslash \mathbb{Q} \cup \mathbb{Q} \\
		\implies |\mathbb{R}| &= |\mathbb{R}\backslash \mathbb{Q} \cup \mathbb{Q}|.
	\end{align}
	We know that $\mathbb{Q}$ is countably infinite, so $|\mathbb{Q}| = |\mathbb{N}|$. We also assumed that $\mathbb{R}\backslash\mathbb{Q}$ is countable, so $|\mathbb{R}\backslash \mathbb{Q}|= |\mathbb{N}|$. Then by part \textbf{(a)}, we have that $|\mathbb{R}| = |\mathbb{N}|$, but the reals are uncountable, so contradiction $(\Rightarrow \Leftarrow)$.
	
	Hence $\mathbb{R}\backslash\mathbb{Q}$ must be uncountable.
\end{proof}

\section*{Problem 4}
\begin{proof}
	($\Rightarrow$) Suppose $a_0 = \sup{A}$. Then for any other upper bound $b \in \mathbb{R}$ of $A$, $a_0 \leq b$. Also, for any $a\in A$, $a \leq a_0$.
	
	Let $\epsilon>0$. Then $a_0 + \epsilon > a_0 \implies a_0 - \epsilon < a_0$, but $a_0 = \sup{A}$ so $a_0 - \epsilon \neq \sup{A}$.  Then $\exists a \in A$ such that $a>a_0 - \epsilon$.
	
	($\Leftarrow$) Suppose $a_0$ is an upper bound for $A\subset\mathbb{R}$, and that for every $\epsilon > 0$ $\exists a\in A$ such that $a_0 - \epsilon < a$. Since $A$ is bounded above by $a_0$, then $\forall a \in A$, $a\leq a_0$. Then for every $\epsilon>0$,
	\begin{align}
		&\implies a-\epsilon \leq a_0 - \epsilon < a \\
		&\implies a \leq a_0 < a + \epsilon. 
	\end{align}
	But $a_0$ is an upper bound for $A$, so we have shown that for any $\epsilon>0$, $\exists a \in A$ such that $a+\epsilon \notin A$ and $a_0 < a + \epsilon$. Thus $a_0$ is the smallest upper bound for $A$, which means that $a_0=\sup{A}$ by definition.
\end{proof}

\section*{Problem 5}
\subsection*{(a)}
\begin{proof}
	(i) Let $\epsilon>0$. Then $a-\epsilon < a \implies a-\epsilon \in (-\infty, a)$.
	Since $(-\infty, a)$ is not bounded below, $\exists x \in (-\infty, a)$ such that $-\infty < x < a-\epsilon$, i.e. $-\infty < x+\epsilon<a$. 
	
	Also, for $\epsilon>0$ and $x\in(-\infty,a)$, it holds that $-\infty<x-\epsilon<a$. So
	\begin{equation}
		(x-\epsilon, x+\epsilon) \subset (-\infty,a).
	\end{equation}
	Therefore, $(-\infty, a)$ is open.
	
	(ii) $\forall x \in (a,b)$, we have that $b-x>0$ and $x-a>0$. Let $\epsilon = \frac{1}{2}\min\{x-a, b-x\}>0$, and let $y\in\mathbb{R}$.
	
	If $y\in(x-\epsilon, x+\epsilon)$, then $-\epsilon<y-x<\epsilon$. But $\epsilon < b-x \implies y-x < b-x$, i.e. $y<b$.
	
	Also, $\epsilon < x-a\implies -\epsilon>a-x$, so $y-x>a-x$, i.e. $y>a$. Thus, $a<y<b$ $\forall y\in(x-\epsilon, x+\epsilon)$.
	
	Therefore, $(a,b)$ is open.
	
	(iii) Let $x\in (b, \infty)$. Then $x>b$. Let $\epsilon = \frac{x-b}{2}>0$, and let $y\in\mathbb{R}$. 
	
	If $y\in(x-\epsilon, x+\epsilon)$, then $-\epsilon<y-x<\epsilon$. Since $\epsilon<x-b \implies -\epsilon>b-x$, 
	\begin{align}
		&\implies b-x<y-x<x-b \\
		&\implies b<y<2x-b< \infty \\
		&\implies y \in (b, \infty).
	\end{align}
	Therefore, $(b, \infty)$ is closed. 
\end{proof}

\subsection*{(b)}
\begin{proof}
	Suppose $U_{\lambda}\subset \mathbb{R}$ is open $\forall \lambda \in \Lambda$. 
	
	Take any $x\in\bigcup_{\lambda\in\Lambda}U_{\lambda}$. Then $x\in U_{\lambda}$ for at least one $\lambda\in\Lambda$. But since the $U_{\lambda}$ are open, then $\exists \epsilon >0$ such that $(x-\epsilon, x+\epsilon)\subset U_{\lambda}$. 
	
	But this must hold for every $x\in\bigcup_{\lambda\in\Lambda}U_{\lambda}$, i.e. for every such $x$, $\exists \epsilon>0$ such that
	\begin{equation}
		(x-\epsilon, x+\epsilon)\subset \bigcup_{\lambda\in\Lambda}U_{\lambda}.
	\end{equation} 
	Therefore, the union must also be open.
\end{proof}

\subsection*{(c)}
\begin{proof}
	Suppose $U_m$ is open $\forall m \in \{1, ..., n\}$. Choose any $x\in\bigcap_{m=1}^{n}U_m$. Then if $x$ is in the intersection of the sets, it must be in each of the sets, i.e. \\$x\in U_1, U_2, ..., U_n$.
	
	Since the $U_m$ are open, then $\exists \epsilon>0$ such that $(x-\epsilon, x+\epsilon)\subset U_m$ $\forall m \in \{1, ..., m\}$. But this must hold for every $x$ in the intersection, i.e. $\forall x \in \bigcap_{m=1}^n U_m$, $\exists \epsilon>0$ such that
	\begin{equation}
		(x-\epsilon, x+\epsilon)\subset \bigcap_{m=1}^{n} U_m.
	\end{equation}  
	Therefore, the intersection must also be open. 
\end{proof}

\subsection*{(d)}
\underline{Claim:} The set of rationals $\mathbb{Q} \subset \mathbb{R}$ is NOT open.
\begin{proof}
	(By contradiction). Suppose instead that $\mathbb{Q}$ is open. 
	
	Then for every $q \in \mathbb{Q}$, $\exists \epsilon>0$ such that
	\begin{equation}
		(q-\epsilon, q+\epsilon) \subset \mathbb{Q}.
	\end{equation}
	Since $q-\epsilon, q+\epsilon \in \mathbb{R}$ and $q-\epsilon < q+\epsilon$, then $\exists i \in \mathbb{R} \backslash \mathbb{Q}$ such that 
	\begin{equation}
		q-\epsilon<i<q+\epsilon,
	\end{equation}
	as proven in problem 1. Then since there is an irrational number $i \in (q-\epsilon, q+\epsilon)$, this implies 
	\begin{equation}
		(q-\epsilon, q+\epsilon) \not\subset \mathbb{Q}, \quad (\Rightarrow \Leftarrow)
	\end{equation}
	which is a contradiction to the initial assumption. 
	
	Therefore, $\mathbb{Q}$ is not open. 
\end{proof}

\end{document}