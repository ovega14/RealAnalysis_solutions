\documentclass{article}
\usepackage[utf8]{inputenc}
\usepackage[english]{babel}
\usepackage[]{amsthm} %lets us use \begin{proof}
\usepackage[]{amssymb} %gives us the character \varnothing
\usepackage[]{amsmath}
\usepackage[parfill]{parskip} %avoid indent when skipping lines
\usepackage[toc,page]{appendix}
\usepackage{mathtools}
\usepackage{hyperref}
\hypersetup{
	colorlinks=true,
	linkcolor=blue,
	filecolor=magenta,      
	urlcolor=cyan,
	pdftitle={18.100A-ps9},
	pdfpagemode=FullScreen,
}
%\urlstyle{same}
\newcommand{\R}{\mathbb{R}} %the real numbers
\newcommand{\N}{\mathbb{N}} %the natural numbers
\newcommand{\M}{\mathcal{M}} %set of all Lebesgue-measurable sets
\newcommand{\Q}{\mathbb{Q}} % the rational numbers

\title{18.100A Assignment 9}
\author{Octavio Vega}
\date\today

\begin{document}
\maketitle

\section*{Problem 1}
\begin{proof}
	We have that $\forall x \in \R$, $|\arctan(x)| < \frac{\pi}{2}$, i.e.
	\begin{equation}
		\arctan(x) \in \left(-\frac{\pi}{2}, \frac{\pi}{2}\right),
	\end{equation}
	which is an open set. So $\forall |y| < \frac{\pi}{2}$, $\exists \epsilon > 0$ such that $(y - \epsilon, y + \epsilon) \subset (-\frac{\pi}{2}, \frac{\pi}{2})$. Thus for every such $y$, we can always find a $y_0 < y$ and $y_1 > y$ inside this open set. This means that there is no $x_1$ such that $\arctan(x_1) \geq \arctan(x)$ nor an $x_0$ such that $\arctan(x_0) \leq \arctan(x)$ $\forall x$.
	
	Hence $f(x) = \arctan(x)$ does not achieve an absolute minimum or maximum.
\end{proof}
%%%%%%%%%%%%%%%%%%%%%%%%%%%%%%%%%%%%%%%%%%%%%%%%%%%%%%%%%%%%%%%%%%%%%%%%%%%%%%%%%%%%%%%%%%%%%%%%%%%%%%%%%%%%%%%%%%%%%%%%%%%%%%%%%%%%%%%%%%%%
\section*{Problem 2}
\begin{proof}
	Let $x, y \in (c, \infty)$. Choose $L = \frac{1}{c^2}$. Then
	\begin{align}
		|f(y) - f(x)| &= \big|\frac{1}{y} - \frac{1}{x}\big| \\
		&= \frac{|x - y|}{xy} \\
		&< \frac{|x - y|}{c^2} \\
		&= L |x - y|.
	\end{align}
	Therefore $f(x) = \frac{1}{x}$ is Lipschitz continuous.
\end{proof}
%%%%%%%%%%%%%%%%%%%%%%%%%%%%%%%%%%%%%%%%%%%%%%%%%%%%%%%%%%%%%%%%%%%%%%%%%%%%%%%%%%%%%%%%%%%%%%%%%%%%%%%%%%%%%%%%%%%%%%%%%%%%%%%%%%%%%%%%%%%%
\end{document}