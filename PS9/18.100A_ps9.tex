\documentclass{article}
\usepackage[utf8]{inputenc}
\usepackage[english]{babel}
\usepackage[]{amsthm} %lets us use \begin{proof}
\usepackage[]{amssymb} %gives us the character \varnothing
\usepackage[]{amsmath}
\usepackage[parfill]{parskip} %avoid indent when skipping lines
\usepackage[toc,page]{appendix}
\usepackage{mathtools}
\usepackage{hyperref}
\hypersetup{
	colorlinks=true,
	linkcolor=blue,
	filecolor=magenta,      
	urlcolor=cyan,
	pdftitle={18.100A-ps9},
	pdfpagemode=FullScreen,
}
%\urlstyle{same}
\newcommand{\R}{\mathbb{R}} %the real numbers
\newcommand{\N}{\mathbb{N}} %the natural numbers
\newcommand{\M}{\mathcal{M}} %set of all Lebesgue-measurable sets
\newcommand{\Q}{\mathbb{Q}} % the rational numbers

\title{18.100A Assignment 9}
\author{Octavio Vega}
\date\today

\begin{document}
\maketitle

\section*{Problem 1}
\begin{proof}
	We have that $\forall x \in \R$, $|\arctan(x)| < \frac{\pi}{2}$, i.e.
	\begin{equation}
		\arctan(x) \in \left(-\frac{\pi}{2}, \frac{\pi}{2}\right),
	\end{equation}
	which is an open set. So $\forall |y| < \frac{\pi}{2}$, $\exists \epsilon > 0$ such that $(y - \epsilon, y + \epsilon) \subset (-\frac{\pi}{2}, \frac{\pi}{2})$. Thus for every such $y$, we can always find a $y_0 < y$ and $y_1 > y$ inside this open set. This means that there is no $x_1$ such that $\arctan(x_1) \geq \arctan(x)$ nor an $x_0$ such that $\arctan(x_0) \leq \arctan(x)$ $\forall x$.
	
	Hence $f(x) = \arctan(x)$ does not achieve an absolute minimum or maximum.
\end{proof}
%%%%%%%%%%%%%%%%%%%%%%%%%%%%%%%%%%%%%%%%%%%%%%%%%%%%%%%%%%%%%%%%%%%%%%%%%%%%%%%%%%%%%%%%%%%%%%%%%%%%%%%%%%%%%%%%%%%%%%%%%%%%%%%%%%%%%%%%%%%%
\section*{Problem 2}
\begin{proof}
	Let $x, y \in (c, \infty)$. Choose $L = \frac{1}{c^2}$. Then
	\begin{align}
		|f(y) - f(x)| &= \Big|\frac{1}{y} - \frac{1}{x}\Big| \\
		&= \frac{|x - y|}{xy} \\
		&< \frac{|x - y|}{c^2} \\
		&= L |x - y|.
	\end{align}
	Therefore $f(x) = \frac{1}{x}$ is Lipschitz continuous.
\end{proof}
%%%%%%%%%%%%%%%%%%%%%%%%%%%%%%%%%%%%%%%%%%%%%%%%%%%%%%%%%%%%%%%%%%%%%%%%%%%%%%%%%%%%%%%%%%%%%%%%%%%%%%%%%%%%%%%%%%%%%%%%%%%%%%%%%%%%%%%%%%%%
\section*{Problem 3}
\begin{proof}
	Let $\delta > 0$ and choose $\epsilon_0 = |\sin(\delta)|$. Choose $x = \frac{1}{2\pi k + \delta}$ and $c = \frac{1}{2 \pi k}$ for some $k \in \N$. Then
	\begin{align}
		|x - c| &= \Big|\frac{1}{2 \pi k} - \frac{1}{2 \pi k}\Big| \\
		&= \Big|\frac{2 \pi k - (2 \pi k + \delta)}{2 \pi k (2 \pi k + \delta)}\Big| \\
		&= \frac{\delta}{4 \pi^2 k^2 + 2 \pi k \delta} \\
		&< \delta.
	\end{align}
	We also have
	\begin{align}
		|f(x) - f(c)| &= |\sin(2 \pi k + \delta) - \sin(2 \pi k)| \\
		&= |\sin(2 \pi k)\cos(\delta) + \cos(2 \pi k)\sin(\delta) - \sin(2 \pi k)| \\
		&= |\sin(\delta)| \\
		&= \epsilon_0.
	\end{align}
	Hence, $f(x) = \sin\left(\frac{1}{x}\right)$ is not uniformly continuous.
\end{proof}
%%%%%%%%%%%%%%%%%%%%%%%%%%%%%%%%%%%%%%%%%%%%%%%%%%%%%%%%%%%%%%%%%%%%%%%%%%%%%%%%%%%%%%%%%%%%%%%%%%%%%%%%%%%%%%%%%%%%%%%%%%%%%%%%%%%%%%%%%%%%
\section*{Problem 4}
\begin{proof}
	Suppose $f: S \to \R$ is Lipschitz continuous on $S$. Then $\exists L \geq 0$ such that $\forall x, y \in S$, $|f(x) - f(y)| \leq L |x - y|$.
	
	Let $\epsilon > 0$. Choose $\delta = \frac{\epsilon}{L}$. If $|x - y| < \delta$, then
	\begin{align}
		|f(x) - f(y)| &\leq L|x - y| \\
		& < L \delta \\
		&= \epsilon.
	\end{align}
	Thus $f$ is uniformly continuous on $S$. 
\end{proof}
%%%%%%%%%%%%%%%%%%%%%%%%%%%%%%%%%%%%%%%%%%%%%%%%%%%%%%%%%%%%%%%%%%%%%%%%%%%%%%%%%%%%%%%%%%%%%%%%%%%%%%%%%%%%%%%%%%%%%%%%%%%%%%%%%%%%%%%%%%%%
\section*{Problem 5}
(a) \begin{proof}
	Let $x, y \in \R$. Choose $L = 1$. Then
	\begin{align}
		|f(x) - f(y)| &= |\cos(x) - \cos(y)| \\
		&= \Big|2 \sin\left(\frac{x + y}{2}\right) \sin\left(\frac{x - y}{2}\right)\Big| \\
		& \leq 2 \Big|\sin\left(\frac{x - y}{2}\right)\Big| \\
		& \leq 2 \Big|\frac{x - y}{2}\Big| \\
		&= |x - y|.
	\end{align}
	Therefore $f(x) = \cos(x)$ is Lipschitz continuous on $\R$.
\end{proof}

(b) \begin{proof}
	(1) Let $\epsilon > 0$. Choose $\delta = c^{\frac{2}{3}} \epsilon$. Then $\forall x, c \in [0, 1]$, we have
	\begin{align}
		|f(x) - f(c)| &= |x^{\frac{1}{3}} - c^{\frac{1}{3}}| \\
		&= \frac{|x - c|}{|x^{\frac{2}{3}} + x^{\frac{1}{3}}c^{\frac{1}{3}} + c^{\frac{2}{3}}|} \\
		&< \frac{\delta}{c^{\frac{2}{3}}} \\
		&= \epsilon.
	\end{align}
	Thus, $f(x) = x^{\frac{1}{3}}$ is uniformly continuous on $[0, 1]$.
	
	(2) (By contradiction.)
	
	Suppose $f$ is Lipschitz continuous on $[0, 1]$. Then $\forall x, y \in [0, 1]$, $\exists L \geq 0$ such that $|x^{\frac{1}{3}} - y^{\frac{1}{3}}| \leq L |x - y|$.
	
	Choose $y = 0$. Then $|x^{\frac{1}{3}}| \leq L |x|$, i.e. $\frac{1}{x^{\frac{2}{3}}} \leq L$. Taking $x \to 0$ on both sides, this implies that $\lim\limits_{x \to 0} \frac{1}{x^{\frac{2}{3}}}$ exists and is finite. But we know that this limit does not exist, so we have arrived at a contradiction.
	
	Therefore $f(x) = x^{\frac{1}{3}}$ is not Lipschitz continuous on $[0, 1]$.
\end{proof}
%%%%%%%%%%%%%%%%%%%%%%%%%%%%%%%%%%%%%%%%%%%%%%%%%%%%%%%%%%%%%%%%%%%%%%%%%%%%%%%%%%%%%%%%%%%%%%%%%%%%%%%%%%%%%%%%%%%%%%%%%%%%%%%%%%%%%%%%%%%%



\end{document}