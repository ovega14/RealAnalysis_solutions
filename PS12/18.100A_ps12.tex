\documentclass{article}
\usepackage[utf8]{inputenc}
\usepackage[english]{babel}
\usepackage[]{amsthm} %lets us use \begin{proof}
\usepackage[]{amssymb} %gives us the character \varnothing
\usepackage[]{amsmath}
\usepackage[parfill]{parskip} %avoid indent when skipping lines
\usepackage[toc,page]{appendix}
\usepackage{mathtools}
\usepackage{hyperref}
\hypersetup{
	colorlinks=true,
	linkcolor=blue,
	filecolor=magenta,      
	urlcolor=cyan,
	pdftitle={18.100A-ps12},
	pdfpagemode=FullScreen,
}
%\urlstyle{same}
\newcommand{\R}{\mathbb{R}} %the real numbers
\newcommand{\N}{\mathbb{N}} %the natural numbers
\newcommand{\M}{\mathcal{M}} %set of all Lebesgue-measurable sets
\newcommand{\Q}{\mathbb{Q}} % the rational numbers

\title{18.100A Assignment 12}
\author{Octavio Vega}
\date\today

\begin{document}
\maketitle
	
\section*{Problem 1}
\subsection*{(a)}
\begin{proof}
	Suppose $\exists c \in [a, b]$ such that $f(c) > 0$. Since $f$ is continuous, $\exists \delta > 0$ such that if $|x - c| < \delta$, then $|f(x) - f(x)| < \frac{f(c)}{2}$, i.e. $\frac{f(c)}{2} < f(x)$. We compute
	\begin{align}
		0 &> \int_a^b f \\
		&> \int_a ^b \frac{f(c)}{2} \\
		&= \frac{f(c)}{2}(b - a) \\
		&> 0,
	\end{align}
	i.e. $0 > 0$ $(\Rightarrow \Leftarrow)$, which is cleary a contradiction.
	
	Therefore $f(x) = 0$ $\forall x \in [a, b]$.  
\end{proof}

\subsection*{(b)}
\begin{proof}
	Let $E = \int_a ^b (u')^2 \mathrm{d}x$. Then $E \geq 0$ since $(u')^2 \geq 0$. Using integration by parts, we have
	\begin{align}
		E &= \int_a ^b u' u' \mathrm{d}x \\
		&= u u' \big |_a ^b - \int_a ^b u u'' \mathrm{d}x \\
		&= u'(b) u(b) - u'(a) u(a) - \int_a ^b u (V u) \mathrm{d}x \\
		&= - \int_a ^b V u^2 \mathrm{d}x.
	\end{align}
	But $V(x) \geq 0$ and $u^2 \geq 0$, so $-(Vu^2) \leq 0$, hence $E \leq 0$. Thus $E = 0$, which must mean that 
	\begin{equation}
		\int_a ^b (u')^2 \mathrm{d}x = 0,
	\end{equation}
	and by part \textbf{(a)}, this implies that $(u')^2 = 0$  $\forall x \in [a, b]$. Hence $u'(x) = 0$ for all $x$, and since $u(a) = 0$, then $u$ remains constant at $0$; i.e. $u = 0$ everywhere.
\end{proof}
%%%%%%%%%%%%%%%%%%%%%%%%%%%%%%%%%%%%%%%%%%%%%%%%%%%%%%%%%%%%%%%%%%%%%%%%%%%%%%%%%%%%%%%%%%%%%%%%%%%%%%%%%%%%%%%%%%%%%%%%%%%%%%%%%%%%%%%%%%%%
\section*{Problem 2}
We compute:
\begin{align}
	\int_{-x}^x e^{s^2} \mathrm{d}s &= \int_{-x}^{0}e^{s^2} \mathrm{d}s + \int_{0}^{x}e^{s^2} \mathrm{d}s \\
	&= \int_{0}^{x}e^{s^2} \mathrm{d}s - \int_{0}^{-x}e^{s^2} \mathrm{d}s.
\end{align}
Differentiating, we get
\begin{align}
	\frac{\mathrm{d}}{\mathrm{d}x}\left(\int_{-x}^x e^{s^2} \mathrm{d}s\right) &= \frac{\mathrm{d}}{\mathrm{d}x}\left(\int_{0}^{x}e^{s^2}\right) - \frac{\mathrm{d}}{\mathrm{d}x}\left(\int_{0}^{-x}e^{s^2}\right) \\
	&= e^{x^2} + e^{x^2}.
\end{align}
Thus $\frac{\mathrm{d}}{\mathrm{d}x}\left(\int_{-x}^x e^{s^2} \mathrm{d}s\right) = 2e^{x^2}$.
%%%%%%%%%%%%%%%%%%%%%%%%%%%%%%%%%%%%%%%%%%%%%%%%%%%%%%%%%%%%%%%%%%%%%%%%%%%%%%%%%%%%%%%%%%%%%%%%%%%%%%%%%%%%%%%%%%%%%%%%%%%%%%%%%%%%%%%%%%%%
\section*{Problem 3}
\begin{proof}
	Define $G(x) = \int_a ^x f(t) \mathrm{d}t$. By the fundamental theorem of calculus, $G$ is continuous on $[a, b]$. Note that $G(x) = 0$  $\forall x \in \Q \cap [a, b]$. We now claim that $G(x) = 0$ on $[a, b]$. 
	
	Suppose $\exists c \in [a, b]$ such that $G(c) \neq 0$. For some $x \in [a, b]$, let $\epsilon = \frac{|G(x)|}{2}$ and let $\delta > 0$. Then $\forall x$ such that $|x - c| < \delta$, we have $|G(x) - G(c)| < \epsilon$ since $G$ is continuous. $\exists c \in [a, b] \cap \Q$ such that $|x - c| < \delta$. But then $|G(x) - G(c)| = |G(x)| > \frac{|G(x)|}{2} = \epsilon$, which is a contradiction, since we assumed $G$ to be continuous. Thus $G(x) = 0$ on $[a, b]$, which proves the claim.
	
	Thus, $G(x) = \int_a ^x f(t) \mathrm{d}t = 0$ on $[a, b]$. Since $G$ is constant, then $G' = 0$ on $[a, b]$. By the fundamental theorem of calculus, $G'(x) = f(x) = 0$ $\forall x \in [a, b]$, and we are done.
\end{proof}
%%%%%%%%%%%%%%%%%%%%%%%%%%%%%%%%%%%%%%%%%%%%%%%%%%%%%%%%%%%%%%%%%%%%%%%%%%%%%%%%%%%%%%%%%%%%%%%%%%%%%%%%%%%%%%%%%%%%%%%%%%%%%%%%%%%%%%%%%%%%
\section*{Problem 4}
\subsection{(a)}
Let $f_n(x) = \frac{e^{\frac{x}{n}}}{n}$ for each $n \in \N$. Then by continuity, we have
\begin{align}
	\lim\limits_{n \to \infty} f_n(x) &= \lim\limits_{n \to \infty} e^{\frac{x}{n}} \cdot \lim\limits_{n \to \infty} \frac{1}{n} \\
	&= e^{x \lim\limits_{n \to \infty}\frac{1}{n}} \cdot \lim\limits_{n \to \infty} \frac{1}{n} \\
	&= e^0 \cdot 0 \\
	&= 0.
\end{align}
Therefore $f_n \to 0$ pointwise.

\subsection*{(b)}
Let $M \in \N$. Choose $\epsilon_0 = 1$, $x = n \ln(n)$, and $n = M$. Then
\begin{align}
	|f_n(x) - f(x)| &= \left|\frac{e^{\frac{x}{n}}}{n}\right| \\
	&= \frac{e^{\ln(M)}}{M} \\
	&= \frac{M}{M} \\
	&= 1 \\
	&= \epsilon_0.
\end{align}
Therefore the limit is NOT uniform on $\R$.

\subsection*{(c)}
Let $\epsilon > 0$. Choose $M = \frac{1}{\log(\epsilon)}$. Then $\forall x \in [0, 1]$ and $\forall n \geq M$, we have
\begin{align}
	|f_n(x) - 0| &= \left|\frac{e^\frac{x}{n}}{n}\right| \\
	&< \frac{e^\frac{1}{n}}{n} \\
	&< e^\frac{1}{n} \\
	&< e^{\log(\epsilon)} \\
	&= \epsilon.
\end{align}
Therefore the limit is uniform on $[0, 1]$.
%%%%%%%%%%%%%%%%%%%%%%%%%%%%%%%%%%%%%%%%%%%%%%%%%%%%%%%%%%%%%%%%%%%%%%%%%%%%%%%%%%%%%%%%%%%%%%%%%%%%%%%%%%%%%%%%%%%%%%%%%%%%%%%%%%%%%%%%%%%%
\section*{Problem 5}

\end{document}