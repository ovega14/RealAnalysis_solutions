\documentclass{article}
\usepackage[utf8]{inputenc}
\usepackage[english]{babel}
\usepackage[]{amsthm} %lets us use \begin{proof}
\usepackage[]{amssymb} %gives us the character \varnothing
\usepackage[]{amsmath}
\usepackage[parfill]{parskip} %avoid indent when skipping lines
\usepackage{hyperref}
\hypersetup{
	colorlinks=true,
	linkcolor=blue,
	filecolor=magenta,      
	urlcolor=cyan,
	pdftitle={18.100A-ps4},
	pdfpagemode=FullScreen,
}
%\urlstyle{same}

\title{18.100A Assignment 4}
\author{Octavio Vega}
\date\today

\begin{document}
\maketitle

\section*{Problem 1}
\subsection*{(a)}
\begin{proof}
	Define the complement of $[a,b]$ via
	\begin{equation}
		[a,b]^c := \{x\in\mathbb{R} \;|\; x<a, \; x>b\}.
	\end{equation}
	We can write this complement as the union of two sets:
	\begin{align}
		[a,b]^c &= \{x\in\mathbb{R} \;|\; x<a \} \cup \{x\in\mathbb{R} \;|\; x>b \} \\
		&= (-\infty, a) \cup (b, \infty).
	\end{align}
	Both the sets $(-\infty, a)$ and $(b, \infty)$ are open, as proved in \href{https://github.com/ovega14/RealAnalysis_solutions/blob/main/PS3/18.100A_ps3.pdf}{PS3.5a}. We also proved that the union of open sets is open. Thus, $[a,b]^c$ is open.
	
	Therefore, we conclude that $[a,b]$ is closed.
\end{proof}

\subsection*{(b)}
\underline{Claim:} The set $\mathbb{Z}\subset\mathbb{R}$ is closed.
\begin{proof}
	Consider the complement of the integers in the real numbers, $\mathbb{Z}^c = \mathbb{R}\backslash\mathbb{Z}$. We may write this complement as a union of open sets, where each of the open sets represents the set of numbers between (but not including) consecutive integers:
	\begin{equation}
		\mathbb{Z}^c = \bigcup\limits_{n\in\mathbb{Z}}(n, n+1).
	\end{equation}
	Since the sets being unioned are all open, then so is the union, i.e. $\mathbb{Z}^c$ is open.
	
	Thus, $\mathbb{Z}$ is closed.
\end{proof}

\subsection*{(c)}
\underline{Claim:} The set of rationals $\mathbb{Q}\subset\mathbb{R}$ is not closed.
\begin{proof}
	The complement of the rational numbers $\mathbb{Q}$ in the reals is the set of irrationals:
	\begin{equation}
		\mathbb{Q}^c = \mathbb{R} \backslash \mathbb{Q}.
	\end{equation}
	Let $i \in \mathbb{Q}^c$. In class, we proved the density of $\mathbb{Q}$ in $\mathbb{R}$. Additionally, in \href{https://github.com/ovega14/RealAnalysis_solutions/blob/main/PS3/18.100A_ps3.pdf}{PS3.1}, we proved the density of the irrationals in $\mathbb{R}$. It follows that $\exists q, r \in \mathbb{Q}$ such that $q<i<r$.
	
	Let $\epsilon>0$. Since $i-\epsilon, i+\epsilon \in \mathbb{R}$, then $\exists p\in\mathbb{Q}$ such that $i-\epsilon<p<i+\epsilon$. This implies that $p\in(q,r)$, but $p \in \mathbb{Q} \implies p\notin \mathbb{Q}^c$, so $\mathbb{Q}^c$ is not open.
	
	Therefore, $\mathbb{Q}$ is not closed.
\end{proof}
%%%%%%%%%%%%%%%%%%%%%%%%%%%%%%%%%%%%%%%%%%%%%%%%%%%%%%%%%%%%%%%%%%%%%%%%%%%%%%%%%%%%%%%%%%%%%%%%%%%%%%%%%%%%%%%%%%%%%%%%%%%%%%%%%%%%%%%%%%%%
\section*{Problem 2}
\subsection*{(a)}
\begin{proof}
	Let $x \notin \bigcap_{\lambda \in \Lambda} F_{\lambda}$. Then $x\in\left(\bigcap_{\lambda\in\Lambda}F_{\lambda}\right)^c$, so
	\begin{equation}
		x \in \bigcup\limits_{\lambda \in \Lambda}F_{\lambda}^c.
	\end{equation}
	So for at least one $\lambda \in \Lambda$, $x\in F_{\lambda}^c$.
	
	Since $F_{\lambda}$ is closed, then $F_{\lambda}^c$ is open. Thus $\exists \epsilon>0$ such that $(x-\epsilon, x+\epsilon)\subset F_{\lambda}^c$. But since this must hold for arbitrary $x$, then it holds for every \\$x\in \left(\bigcap_{\lambda\in\Lambda}F_{\lambda}\right)^c$. 
	
	Hence, $\left(\bigcap_{\lambda\in\Lambda}F_{\lambda}\right)^c$ is open $\implies \bigcap_{\lambda\in\Lambda}F_{\lambda}$ is closed.
\end{proof}

\subsection*{(b)}
\begin{proof}
	Let $x \notin \bigcup_{m=1}^n F_m$. Then
	\begin{equation}\label{union_complement}
		x \in \left(\bigcup\limits_{m=1}^n F_m\right)^c \implies x \in \bigcap\limits_{m=1}^n F_m^c.
	\end{equation}
	So $\forall m \in \{1, ..., n\}$, $x\in F_m^c$. Similarly,
	\begin{equation}\label{intersect_complement}
		x \in \bigcap\limits_{m=1}^n F_m^c \implies x \in \left(\bigcup\limits_{m=1}^n F_m\right)^c. 
	\end{equation}
	[\textit{Intuition}: If $x$ is in the complement of the union of several sets, then it can't be in any of them individually (it must simultaneously be in none of them), which means that it must be in the intersection of the complements of each of the sets. Likewise, if $x$ is in the intersection of the complements of several sets, then it must be in none of the individual sets, so it has to be outside the union of all the sets.]
	
	Equations \eqref{union_complement} and \eqref{intersect_complement} imply $\left(\bigcup_{m=1}^n F_m\right)^c \subseteq \bigcap_{m=1}^n F_m^c$ and $\bigcap_{m=1}^n F_m^c \subseteq \left(\bigcup_{m=1}^n F_m\right)^c$, respectively. Then
	\begin{equation}\label{sets_equal}
		\left(\bigcup\limits_{m=1}^n F_m\right)^c = \bigcap\limits_{m=1}^n F_m^c.
	\end{equation}
	Since $F_m$ is closed, then $F_m^c$ is open. Then the intersection $\bigcap_{m=1}^n F_m^c$ must be open, which implies by \eqref{sets_equal} that $\left(\bigcup_{m=1}^n F_m\right)^c$ is open.
	
	Therefore, $\bigcup_{m=1}^n F_m$ is closed.
\end{proof}
%%%%%%%%%%%%%%%%%%%%%%%%%%%%%%%%%%%%%%%%%%%%%%%%%%%%%%%%%%%%%%%%%%%%%%%%%%%%%%%%%%%%%%%%%%%%%%%%%%%%%%%%%%%%%%%%%%%%%%%%%%%%%%%%%%%%%%%%%%%%
\section*{Problem 3}
\begin{proof}
	(By contradiction). Suppose instead that $x \in F^c$. Since $F$ is closed, then $F^c$ is open. Then $\exists \epsilon > 0$ such that $(x-\epsilon, x+\epsilon) \subset F^c$. 
	
	Since $\{x_n\}$ converges to $x$, then $\lim_{n\to\infty}x_n = x$. So for every $\epsilon > 0$, $\exists N \in \mathbb{N}$ such that $\forall n > N$, $|x_n-x|<\epsilon$. Then for $n>N$, 
	\begin{equation}
		x-\epsilon < x_n<x+\epsilon.
	\end{equation}
	But then $\forall n > N$, $x_n \in (x-\epsilon, x+\epsilon)\subset F^c$, i.e. $x_n \in F^c$ ($\Rightarrow\Leftarrow$). This is a contradiction because we assumed that the elements of $\{x_n\}_n$ are in $F$. 
	
	Therefore, $x \in F$.
\end{proof}
%%%%%%%%%%%%%%%%%%%%%%%%%%%%%%%%%%%%%%%%%%%%%%%%%%%%%%%%%%%%%%%%%%%%%%%%%%%%%%%%%%%%%%%%%%%%%%%%%%%%%%%%%%%%%%%%%%%%%%%%%%%%%%%%%%%%%%%%%%%%
\end{document}