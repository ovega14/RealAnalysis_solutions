\documentclass{article}
\usepackage[utf8]{inputenc}
\usepackage[english]{babel}
\usepackage[]{amsthm} %lets us use \begin{proof}
\usepackage[]{amssymb} %gives us the character \varnothing
\usepackage[]{amsmath}
\usepackage[parfill]{parskip} %avoid indent when skipping lines

\title{18.100A Assignment 4}
\author{Octavio Vega}
\date\today

\begin{document}
\maketitle

\section*{Problem 1}
\subsection*{(a)}
\begin{proof}
	Define the complement of $[a,b]$ via
	\begin{equation}
		[a,b]^c := \{x\in\mathbb{R} \;|\; x<a, \; x>b\}.
	\end{equation}
	We can write this complement as the union of two sets:
	\begin{align}
		[a,b]^c &= \{x\in\mathbb{R} \;|\; x<a \} \cup \{x\in\mathbb{R} \;|\; x>b \} \\
		&= (-\infty, a) \cup (b, \infty).
	\end{align}
	Both the sets $(-\infty, a)$ and $(b, \infty)$ are open, as proved in assignment 3. We also proved that the union of open sets is open. Thus, $[a,b]^c$ is open.
	
	Therefore, we conclude that $[a,b]$ is closed.
\end{proof}

\subsection*{(b)}
\begin{proof}
	Consider the complement of the integers in the real numbers, $\mathbb{Z}^c = \mathbb{R}\backslash\mathbb{Z}$. We may write this complement as a union of open sets, where each of the open sets represents the set of numbers between (but not including) consecutive integers:
	\begin{equation}
		\mathbb{Z}^c = \bigcup\limits_{n\in\mathbb{Z}}(n, n+1).
	\end{equation}
	Since the sets being unioned are all open, then so is the union, i.e. $\mathbb{Z}^c$ is open.
	
	Thus, $\mathbb{Z}$ is closed.
\end{proof}
\end{document}