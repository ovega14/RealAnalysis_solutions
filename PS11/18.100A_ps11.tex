\documentclass{article}
\usepackage[utf8]{inputenc}
\usepackage[english]{babel}
\usepackage[]{amsthm} %lets us use \begin{proof}
\usepackage[]{amssymb} %gives us the character \varnothing
\usepackage[]{amsmath}
\usepackage[parfill]{parskip} %avoid indent when skipping lines
\usepackage[toc,page]{appendix}
\usepackage{mathtools}
\usepackage{hyperref}
\hypersetup{
	colorlinks=true,
	linkcolor=blue,
	filecolor=magenta,      
	urlcolor=cyan,
	pdftitle={18.100A-ps11},
	pdfpagemode=FullScreen,
}
%\urlstyle{same}
\newcommand{\R}{\mathbb{R}} %the real numbers
\newcommand{\N}{\mathbb{N}} %the natural numbers
\newcommand{\M}{\mathcal{M}} %set of all Lebesgue-measurable sets
\newcommand{\Q}{\mathbb{Q}} % the rational numbers

\title{18.100A Assignment 11}
\author{Octavio Vega}
\date\today

\begin{document}
\maketitle
	
\section*{Problem 1}
\begin{proof}
	Let $f(x) = \frac{1}{1121}x^{1121} + \frac{1}{2021}x^{2021} + x + 1$. We compute the derivative:
	\begin{equation}
		f'(x) = x^{1120} + x^{2020} + 1.
	\end{equation}
	Hence $f'(x) > 0$ $\forall x \in \R$, so $f(x)$ is increasing.
	
	Suppose $f(x)$ has $n$ real roots, where $n > 1$. Then $\exists x_1, x_2, ..., x_n$ such that $f(x_1) = \cdots = f(x_n) = 0$. Since $f(x)$ is polynomial, then $f$ is continuous $\forall x$ and differentiable on $\R$. By Rolle's theorem, $\exists c \in (x_1, x_2)$ such that $f'(c) = 0$, which is a contradiction since $f'(x) > 0$ $\forall x$. Thus $f$ cannot have more than one real root.
	
	Now suppose $f(x)$ has no real roots. Then either $f(x) > 0$ $\forall x$ or $f(x) < 0$ $\forall x$. Choose $x_0 = 1$. Then $f(x_0) = \frac{1}{1121} + \frac{1}{2021} + 2 > 0$. Choose $x^* = -10$. Then $f(x^*) = -\frac{10^{1121}}{1121} - \frac{10^{2021}}{2021} + 2 < 0$.
	Then by continuity, $f(x)$ has at least one real root, which is a contradiction. Thus $f$ must have at least one real root.
	
	So we have the number of real roots $1 \leq n \leq 1$, so $n = 1$.
	
	Therefore $f$ has exactly one real root.
\end{proof}

\end{document}