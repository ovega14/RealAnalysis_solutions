\documentclass{article}
\usepackage[utf8]{inputenc}
\usepackage[english]{babel}
\usepackage[]{amsthm} %lets us use \begin{proof}
\usepackage[]{amssymb} %gives us the character \varnothing
\usepackage[]{amsmath}
\usepackage[parfill]{parskip} %avoid indent when skipping lines
\usepackage[toc,page]{appendix}
\usepackage{mathtools}
\usepackage{hyperref}
\hypersetup{
	colorlinks=true,
	linkcolor=blue,
	filecolor=magenta,      
	urlcolor=cyan,
	pdftitle={18.100A-ps11},
	pdfpagemode=FullScreen,
}
%\urlstyle{same}
\newcommand{\R}{\mathbb{R}} %the real numbers
\newcommand{\N}{\mathbb{N}} %the natural numbers
\newcommand{\M}{\mathcal{M}} %set of all Lebesgue-measurable sets
\newcommand{\Q}{\mathbb{Q}} % the rational numbers

\title{18.100A Assignment 11}
\author{Octavio Vega}
\date\today

\begin{document}
\maketitle
	
\section*{Problem 1}
\begin{proof}
	Let $f(x) = \frac{1}{1121}x^{1121} + \frac{1}{2021}x^{2021} + x + 1$. We compute the derivative:
	\begin{equation}
		f'(x) = x^{1120} + x^{2020} + 1.
	\end{equation}
	Hence $f'(x) > 0$ $\forall x \in \R$, so $f(x)$ is increasing.
	
	Suppose $f(x)$ has $n$ real roots, where $n > 1$. Then $\exists x_1, x_2, ..., x_n$ such that $f(x_1) = \cdots = f(x_n) = 0$. Since $f(x)$ is polynomial, then $f$ is continuous $\forall x$ and differentiable on $\R$. By Rolle's theorem, $\exists c \in (x_1, x_2)$ such that $f'(c) = 0$, which is a contradiction since $f'(x) > 0$ $\forall x$. Thus $f$ cannot have more than one real root.
	
	Now suppose $f(x)$ has no real roots. Then either $f(x) > 0$ $\forall x$ or $f(x) < 0$ $\forall x$. Choose $x_0 = 1$. Then $f(x_0) = \frac{1}{1121} + \frac{1}{2021} + 2 > 0$. Choose $x^* = -10$. Then $f(x^*) = -\frac{10^{1121}}{1121} - \frac{10^{2021}}{2021} + 2 < 0$.
	Then by continuity, $f(x)$ has at least one real root, which is a contradiction. Thus $f$ must have at least one real root.
	
	So we have the number of real roots $1 \leq n \leq 1$, so $n = 1$.
	
	Therefore $f$ has exactly one real root.
\end{proof}
%%%%%%%%%%%%%%%%%%%%%%%%%%%%%%%%%%%%%%%%%%%%%%%%%%%%%%%%%%%%%%%%%%%%%%%%%%%%%%%%%%%%%%%%%%%%%%%%%%%%%%%%%%%%%%%%%%%%%%%%%%%%%%%%%%%%%%%%%%%%
\section*{Problem 2}
\subsection*{(a)}
Let $f(x) = \sin(x)$ and $x_0 = 0$. We compute
\begin{equation}
	P_4(x) = \sum_{k=0}^4 \frac{f^{(k)}(x_0)}{k!}(x - x_0)^k.
\end{equation}
The derivatives are
\begin{align}
	f'(x) = \cos(x) &\implies f'(x_0) = \cos(0) = 1 \\
	f''(x) = -\sin(x) &\implies f''(x_0) = -\sin(0) = 0 \\
	f'''(x) = -\cos(x) &\implies f'''(x_0) = -\cos(0) = -1 \\
	f^{(4)}(x) = \sin(x) &\implies f^{(4)}(x_0) = \sin(0) = 0.
\end{align}
Therefore, the fourth Taylor polynomial is
\begin{equation}
	P_4(x) = x - \frac{1}{3!}x^3.
\end{equation}

\subsection*{(b)}
Let $f(x) = \frac{1}{1-x}$ and $x_0 = -1$. The derivatives are
\begin{align}
	f'(x) = \frac{1}{(1-x)^2} &\implies f'(x_0) = \frac{1}{4} \\
	f''(x) = \frac{2}{(1-x)^3} &\implies f''(x_0) = \frac{1}{4} \\
	f'''(x) = \frac{6}{(1-x)^4} &\implies f'''(x_0) = \frac{3}{8} \\
	f^{(4)}(x) = \frac{24}{(1-x)^5} &\implies f^{(4)}(x_0) = \frac{3}{4}.
\end{align}
Therefore, the fourth Taylor polynomial is
\begin{equation}
	P_4(x) = \frac{1}{2} + \frac{1}{4}(x + 1) + \frac{1}{8}(x + 1)^2 + \frac{1}{16}(x + 1)^3 + \frac{1}{32}(x + 1)^4.
\end{equation}
%%%%%%%%%%%%%%%%%%%%%%%%%%%%%%%%%%%%%%%%%%%%%%%%%%%%%%%%%%%%%%%%%%%%%%%%%%%%%%%%%%%%%%%%%%%%%%%%%%%%%%%%%%%%%%%%%%%%%%%%%%%%%%%%%%%%%%%%%%%%
\section*{Problem 3}
\subsection*{(a)}
We compute using L'Hopital:
\begin{align}
	\lim\limits_{x \to 0} \frac{x - \sin(x)}{x^3} &= \lim\limits_{x \to 0}\frac{1 - \cos(x)}{3x^2} \\
	&= \lim\limits_{x \to 0} \frac{\sin(x)}{6x} \\
	&= \lim\limits_{x \to 0} \frac{\cos(x)}{6} \\
	&= \frac{1}{6}.
\end{align}

\subsection*{(b)}
We proceed again using L'Hopital's rule:
\begin{align}
	\lim\limits_{x \to \frac{\pi}{2}} \frac{1 - \sin(x)}{\left(x - \frac{\pi}{2}\right)^2} &= \lim\limits_{x \to \frac{\pi}{2}} \frac{-\cos(x)}{2 \left(x - \frac{\pi}{2}\right)} \\
	&= -\frac{1}{2}\lim\limits_{x \to \frac{\pi}{2}} \sin(x) \\
	&= -\frac{1}{2}.
\end{align}

\end{document}