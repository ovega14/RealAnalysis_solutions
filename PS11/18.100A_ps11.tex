\documentclass{article}
\usepackage[utf8]{inputenc}
\usepackage[english]{babel}
\usepackage[]{amsthm} %lets us use \begin{proof}
\usepackage[]{amssymb} %gives us the character \varnothing
\usepackage[]{amsmath}
\usepackage[parfill]{parskip} %avoid indent when skipping lines
\usepackage[toc,page]{appendix}
\usepackage{mathtools}
\usepackage{hyperref}
\hypersetup{
	colorlinks=true,
	linkcolor=blue,
	filecolor=magenta,      
	urlcolor=cyan,
	pdftitle={18.100A-ps11},
	pdfpagemode=FullScreen,
}
%\urlstyle{same}
\newcommand{\R}{\mathbb{R}} %the real numbers
\newcommand{\N}{\mathbb{N}} %the natural numbers
\newcommand{\M}{\mathcal{M}} %set of all Lebesgue-measurable sets
\newcommand{\Q}{\mathbb{Q}} % the rational numbers

\title{18.100A Assignment 11}
\author{Octavio Vega}
\date\today

\begin{document}
\maketitle
	
\section*{Problem 1}
\begin{proof}
	Let $f(x) = \frac{1}{1121}x^{1121} + \frac{1}{2021}x^{2021} + x + 1$. We compute the derivative:
	\begin{equation}
		f'(x) = x^{1120} + x^{2020} + 1.
	\end{equation}
	Hence $f'(x) > 0$ $\forall x \in \R$, so $f(x)$ is increasing.
	
	Suppose $f(x)$ has $n$ real roots, where $n > 1$. Then $\exists x_1, x_2, ..., x_n$ such that $f(x_1) = \cdots = f(x_n) = 0$. Since $f(x)$ is polynomial, then $f$ is continuous $\forall x$ and differentiable on $\R$. By Rolle's theorem, $\exists c \in (x_1, x_2)$ such that $f'(c) = 0$, which is a contradiction since $f'(x) > 0$ $\forall x$. Thus $f$ cannot have more than one real root.
	
	Now suppose $f(x)$ has no real roots. Then either $f(x) > 0$ $\forall x$ or $f(x) < 0$ $\forall x$. Choose $x_0 = 1$. Then $f(x_0) = \frac{1}{1121} + \frac{1}{2021} + 2 > 0$. Choose $x^* = -10$. Then $f(x^*) = -\frac{10^{1121}}{1121} - \frac{10^{2021}}{2021} + 2 < 0$.
	Then by continuity, $f(x)$ has at least one real root, which is a contradiction. Thus $f$ must have at least one real root.
	
	So we have the number of real roots $1 \leq n \leq 1$, so $n = 1$.
	
	Therefore $f$ has exactly one real root.
\end{proof}
%%%%%%%%%%%%%%%%%%%%%%%%%%%%%%%%%%%%%%%%%%%%%%%%%%%%%%%%%%%%%%%%%%%%%%%%%%%%%%%%%%%%%%%%%%%%%%%%%%%%%%%%%%%%%%%%%%%%%%%%%%%%%%%%%%%%%%%%%%%%
\section*{Problem 2}
\subsection*{(a)}
Let $f(x) = \sin(x)$ and $x_0 = 0$. We compute
\begin{equation}
	P_4(x) = \sum_{k=0}^4 \frac{f^{(k)}(x_0)}{k!}(x - x_0)^k.
\end{equation}
The derivatives are
\begin{align}
	f'(x) = \cos(x) &\implies f'(x_0) = \cos(0) = 1 \\
	f''(x) = -\sin(x) &\implies f''(x_0) = -\sin(0) = 0 \\
	f'''(x) = -\cos(x) &\implies f'''(x_0) = -\cos(0) = -1 \\
	f^{(4)}(x) = \sin(x) &\implies f^{(4)}(x_0) = \sin(0) = 0.
\end{align}
Therefore, the fourth Taylor polynomial is
\begin{equation}
	P_4(x) = x - \frac{1}{3!}x^3.
\end{equation}

\subsection*{(b)}
Let $f(x) = \frac{1}{1-x}$ and $x_0 = -1$. The derivatives are
\begin{align}
	f'(x) = \frac{1}{(1-x)^2} &\implies f'(x_0) = \frac{1}{4} \\
	f''(x) = \frac{2}{(1-x)^3} &\implies f''(x_0) = \frac{1}{4} \\
	f'''(x) = \frac{6}{(1-x)^4} &\implies f'''(x_0) = \frac{3}{8} \\
	f^{(4)}(x) = \frac{24}{(1-x)^5} &\implies f^{(4)}(x_0) = \frac{3}{4}.
\end{align}
Therefore, the fourth Taylor polynomial is
\begin{equation}
	P_4(x) = \frac{1}{2} + \frac{1}{4}(x + 1) + \frac{1}{8}(x + 1)^2 + \frac{1}{16}(x + 1)^3 + \frac{1}{32}(x + 1)^4.
\end{equation}
%%%%%%%%%%%%%%%%%%%%%%%%%%%%%%%%%%%%%%%%%%%%%%%%%%%%%%%%%%%%%%%%%%%%%%%%%%%%%%%%%%%%%%%%%%%%%%%%%%%%%%%%%%%%%%%%%%%%%%%%%%%%%%%%%%%%%%%%%%%%
\section*{Problem 3}
\subsection*{(a)}
We compute using L'Hopital:
\begin{align}
	\lim\limits_{x \to 0} \frac{x - \sin(x)}{x^3} &= \lim\limits_{x \to 0}\frac{1 - \cos(x)}{3x^2} \\
	&= \lim\limits_{x \to 0} \frac{\sin(x)}{6x} \\
	&= \lim\limits_{x \to 0} \frac{\cos(x)}{6} \\
	&= \frac{1}{6}.
\end{align}

\subsection*{(b)}
We proceed again using L'Hopital's rule:
\begin{align}
	\lim\limits_{x \to \frac{\pi}{2}} \frac{1 - \sin(x)}{\left(x - \frac{\pi}{2}\right)^2} &= \lim\limits_{x \to \frac{\pi}{2}} \frac{-\cos(x)}{2 \left(x - \frac{\pi}{2}\right)} \\
	&= -\frac{1}{2}\lim\limits_{x \to \frac{\pi}{2}} \sin(x) \\
	&= -\frac{1}{2}.
\end{align}
%%%%%%%%%%%%%%%%%%%%%%%%%%%%%%%%%%%%%%%%%%%%%%%%%%%%%%%%%%%%%%%%%%%%%%%%%%%%%%%%%%%%%%%%%%%%%%%%%%%%%%%%%%%%%%%%%%%%%%%%%%%%%%%%%%%%%%%%%%%%
\section*{Problem 4}
\begin{proof}
	(By contradiction.) Assume without loss of generality that $f$ has a local maximum at $c$. Then $\exists \delta > 0$ such that $\forall x \in (a, b)$, if $|x - c| < \delta$ then $f(c) \geq f(x)$.
	
	\underline{Case 1}: $f$ is constant around $c$. Then $\forall |x - c| < \delta$, we have $f'(x) = 0$. By differentiating twice, then we must have $f'''(c) = 0$, which is a contradiction since we assumed $f'''$ to be positive.
	
	So $f$ cannot have a local max at $c$.
	
	\underline{Case 2}: $f$ is not constant around $c$. Then $\forall |x - c| < \delta$, $f(c) > f(x)$. By assumption, $f'''(c) > 0$, so $f''$ is increasing at $c$. Since $f''(c) = 0$, then $f''(x) > 0$ $\forall x > c$. Thus $f'$ is increasing for $x > c$. Since $f'(c)=0$ also by assumption of local maximum, then $f'(x) > 0$ for $x > c$. Thus $f$ is increasing for $x > c$. This means that $f(x) > f(c)$ for $x > c$, which is a contradiction since we assumed that $f(c)$ is a local maximum.
	
	Thus $f$ has no local maximum at $c$. 
	
	To prove that $f$ cannot have a local minimum, repeat the same proof above with $g(x) = -f(x)$, and we are done.
\end{proof}
%%%%%%%%%%%%%%%%%%%%%%%%%%%%%%%%%%%%%%%%%%%%%%%%%%%%%%%%%%%%%%%%%%%%%%%%%%%%%%%%%%%%%%%%%%%%%%%%%%%%%%%%%%%%%%%%%%%%%%%%%%%%%%%%%%%%%%%%%%%%
\section*{Problem 5}
\subsection*{(a)}
We compute
\begin{align}
	\|\underline{x}^{(r)}\| &= \sup_k \left\{x_k^{(r)} - x_{k-1}^{(r)}\right\} \\
	&= \sup_k \left\{(b - a)\frac{k}{r} - (b-a)\frac{k-1}{r}\right\} \\
	&= \sup_k \left\{\frac{b-a}{r}\right\}.
\end{align}
Thus $\|\underline{x}^{(r)}\| = \frac{b-a}{r}$.

\subsection*{(b)}
\begin{proof}
	We write and compute the Riemann sum as
	\begin{align}
		S_f\left(\underline{x}^{(r)}, \underline{\xi}^{(r)}\right) &= \sum_{k=1}^r f\left(\xi_k^{(r)}\right)\left(x_k^{(r)} - x_{k-1}^{(r)}\right) \\
		&= \sum_{k=1}^r f\left(a + (b-a)\frac{k}{r}\right)\left(\frac{b-a}{r}\right) \\
		&= \sum_{k=1}^r \left[\alpha \left(a + (b-a)\frac{k}{r}\right) + \beta\right]\left(\frac{b-a}{r}\right) \\
		&= \sum_{k=1}^r \left[\alpha a + \alpha \frac{bk}{r} - \alpha \frac{ak}{r} + \beta\right] \left(\frac{b-a}{r}\right) \\
		&= \sum_{k=1}^r \left(\frac{\alpha a b}{r} + \frac{\alpha b^2 k}{r^2} - \frac{\alpha a b k}{r^2} + \frac{\beta b}{r} - \frac{\alpha a^2}{r} - \frac{\alpha^2 b k a}{r^2} + \frac{\alpha a^2 k}{r^2} - \frac{\beta a}{r}\right)\\
		&= \alpha a b + \frac{\alpha b^2}{r^2}\sum_{k=1}^r k - \frac{\alpha a b}{r^2}\sum_{k=1}^r k + \beta b - \alpha a^2 - \frac{\alpha^2 b a}{r^2}\sum_{k=1}^r k + \frac{\alpha a^2}{r^2}\sum_{k=1}^2 k - \beta a \\
		&= \alpha a b + \frac{\alpha b^2}{r^2}\frac{r(r+1)}{2} - \frac{\alpha a b}{r^2}\frac{r(r+1)}{2} + \beta b - \alpha a^2 - \frac{\alpha b a}{r^2}\frac{r(r+1)}{2} + \frac{\alpha a^2}{r^2}\frac{r(r+1)}{2} - \beta a \\
		&= \alpha a b + \frac{\alpha b^2}{2} + \frac{\alpha b^2}{2 r} - \frac{\alpha a b}{2} - \frac{\alpha a b}{2 r} + \beta b - \alpha a^2 - \frac{\alpha^2 b a}{2} - \frac{\alpha^2 b a}{2 r} + \frac{\alpha a^2}{2} + \frac{\alpha a^2}{2r} - \beta a.
	\end{align}
	Now sending $r \to \infty$, we find
	\begin{equation}
		\lim\limits_{r \to \infty} S_f \left(\underline{x}^{(r)}, \underline{\xi}^{(r)}\right) = \alpha \left(\frac{b^2 - a^2}{2}\right) + \beta (b - a),
	\end{equation}
	as desired.
\end{proof}

\end{document}