\documentclass{article}
\usepackage[utf8]{inputenc}
\usepackage[english]{babel}
\usepackage[]{amsthm} %lets us use \begin{proof}
\usepackage[]{amssymb} %gives us the character \varnothing
\usepackage[]{amsmath}
\usepackage[parfill]{parskip} %avoid indent when skipping lines
\usepackage[toc,page]{appendix}
\usepackage{mathtools}
\usepackage{hyperref}
\hypersetup{
	colorlinks=true,
	linkcolor=blue,
	filecolor=magenta,      
	urlcolor=cyan,
	pdftitle={18.100A-ps8},
	pdfpagemode=FullScreen,
}
%\urlstyle{same}
\newcommand{\R}{\mathbb{R}} %the real numbers
\newcommand{\N}{\mathbb{N}} %the natural numbers
\newcommand{\M}{\mathcal{M}} %set of all Lebesgue-measurable sets
\newcommand{\Q}{\mathbb{Q}} % the rational numbers

\title{18.100A Assignment 8}
\author{Octavio Vega}
\date\today

\begin{document}
\maketitle

\section*{Problem 1}
\begin{proof}
	Suppose $\lim\limits_{x \to c} f(x) = \lim\limits_{x \to c} h(x)$. Then since $c$ is a cluster point of $S$ and $\forall x \in S$, $f(x) \leq g(x) \leq h(x)$, we have
	\begin{equation}
		\lim\limits_{x \to c} f(x) \leq \lim\limits_{x \to c} g(x) \leq \lim\limits_{x \to c} h(x) = \lim\limits_{x \to c} f(x).
	\end{equation}
	Thus by the squeeze theorem, 
	\begin{equation}
		\lim\limits_{x \to c} g(x) = \lim\limits_{x \to c} f(x) = \lim\limits_{x \to c} h(x),
	\end{equation}
	as desired.
\end{proof}
%%%%%%%%%%%%%%%%%%%%%%%%%%%%%%%%%%%%%%%%%%%%%%%%%%%%%%%%%%%%%%%%%%%%%%%%%%%%%%%%%%%%%%%%%%%%%%%%%%%%%%%%%%%%%%%%%%%%%%%%%%%%%%%%%%%%%%%%%%%%
\section*{Problem 2}
\begin{proof}
	(1) Let $\epsilon > 0$. Choose $\delta = \frac{\epsilon}{2}$. Then if $|x| < \delta$, we have
	\begin{align}
		|f(x) - f(0)| &= |f(x)| \\
		&\leq 2 |x| \\
		& < 2 \delta \\
		&= \epsilon.
	\end{align}
	Therefore, $f$ is continuous at $x = 0$.
	
	(2) Let $\delta > 0$, $\epsilon_0 > 0$. Suppose $|x-1| < \delta$. Let
	\begin{align}
		x_0 = 
		\begin{cases}
			\epsilon_0 \sqrt{2}, \; &\epsilon_0 \in \Q \\
			\epsilon_0, \; &\epsilon_0 \notin \Q.
		\end{cases}
	\end{align}
	Then $x_0 \notin \Q$ $\forall \epsilon_0 > 0$. 
	
	Choose $x = \frac{x_0}{2}$. Then
	\begin{align}
		|f(x) - f(1)| &= |f(x)| \\
		&= 2 |x| \\
		&= x_0 \\
		&\geq \epsilon_0.
	\end{align}
	Therefore, $f$ is discontinuous at $x = 1$.
\end{proof}
	
\end{document}