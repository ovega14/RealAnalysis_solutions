\documentclass{article}
\usepackage[utf8]{inputenc}
\usepackage[english]{babel}
\usepackage[]{amsthm} %lets us use \begin{proof}
\usepackage[]{amssymb} %gives us the character \varnothing
\usepackage[]{amsmath}
\usepackage[parfill]{parskip} %avoid indent when skipping lines
\usepackage[toc,page]{appendix}
\usepackage{mathtools}
\usepackage{hyperref}
\hypersetup{
	colorlinks=true,
	linkcolor=blue,
	filecolor=magenta,      
	urlcolor=cyan,
	pdftitle={18.100A-ps8},
	pdfpagemode=FullScreen,
}
%\urlstyle{same}
\newcommand{\R}{\mathbb{R}} %the real numbers
\newcommand{\N}{\mathbb{N}} %the natural numbers
\newcommand{\M}{\mathcal{M}} %set of all Lebesgue-measurable sets
\newcommand{\Q}{\mathbb{Q}} % the rational numbers

\title{18.100A Assignment 8}
\author{Octavio Vega}
\date\today

\begin{document}
\maketitle

\section*{Problem 1}
\begin{proof}
	Suppose $\lim\limits_{x \to c} f(x) = \lim\limits_{x \to c} h(x)$. Then since $c$ is a cluster point of $S$ and $\forall x \in S$, $f(x) \leq g(x) \leq h(x)$, we have
	\begin{equation}
		\lim\limits_{x \to c} f(x) \leq \lim\limits_{x \to c} g(x) \leq \lim\limits_{x \to c} h(x) = \lim\limits_{x \to c} f(x).
	\end{equation}
	Thus by the squeeze theorem, 
	\begin{equation}
		\lim\limits_{x \to c} g(x) = \lim\limits_{x \to c} f(x) = \lim\limits_{x \to c} h(x),
	\end{equation}
	as desired.
\end{proof}
%%%%%%%%%%%%%%%%%%%%%%%%%%%%%%%%%%%%%%%%%%%%%%%%%%%%%%%%%%%%%%%%%%%%%%%%%%%%%%%%%%%%%%%%%%%%%%%%%%%%%%%%%%%%%%%%%%%%%%%%%%%%%%%%%%%%%%%%%%%%
\section*{Problem 2}
\begin{proof}
	(1) Let $\epsilon > 0$. Choose $\delta = \frac{\epsilon}{2}$. Then if $|x| < \delta$, we have
	\begin{align}
		|f(x) - f(0)| &= |f(x)| \\
		&\leq 2 |x| \\
		& < 2 \delta \\
		&= \epsilon.
	\end{align}
	Therefore, $f$ is continuous at $x = 0$.
	
	(2) Let $\delta > 0$, $\epsilon_0 > 0$. Suppose $|x-1| < \delta$. Let
	\begin{align}
		x_0 = 
		\begin{cases}
			\epsilon_0 \sqrt{2}, \; &\epsilon_0 \in \Q \\
			\epsilon_0, \; &\epsilon_0 \notin \Q.
		\end{cases}
	\end{align}
	Then $x_0 \notin \Q$ $\forall \epsilon_0 > 0$. 
	
	Choose $x = \frac{x_0}{2}$. Then
	\begin{align}
		|f(x) - f(1)| &= |f(x)| \\
		&= 2 |x| \\
		&= x_0 \\
		&\geq \epsilon_0.
	\end{align}
	Therefore, $f$ is discontinuous at $x = 1$.
\end{proof}
%%%%%%%%%%%%%%%%%%%%%%%%%%%%%%%%%%%%%%%%%%%%%%%%%%%%%%%%%%%%%%%%%%%%%%%%%%%%%%%%%%%%%%%%%%%%%%%%%%%%%%%%%%%%%%%%%%%%%%%%%%%%%%%%%%%%%%%%%%%
\section*{Problem 3}
\begin{proof}
	Since $f$ is continuous at $c$, then $\forall \epsilon > 0 $, $\exists \delta > 0$ such that if $|x - c| < \delta$, then $|f(x) - f(c)| < \epsilon$, i.e.
	\begin{equation}
		f(c) - \epsilon < f(x) < f(c) + \epsilon.
	\end{equation}
	Let $\epsilon = \frac{f(c)}{2}$. Then for some $\delta > 0$,
	\begin{equation}
		0 < \frac{f(c)}{2} < f(x) < \frac{3 f(c)}{2}.
	\end{equation}
	Simply choose $\alpha = \delta$, and we are done.
\end{proof}
%%%%%%%%%%%%%%%%%%%%%%%%%%%%%%%%%%%%%%%%%%%%%%%%%%%%%%%%%%%%%%%%%%%%%%%%%%%%%%%%%%%%%%%%%%%%%%%%%%%%%%%%%%%%%%%%%%%%%%%%%%%%%%%%%%%%%%%%%%%
\section*{Problem 4}
\begin{proof}
	Since $h = f$ on $[-1, 0]$ and $f(x)$ is continuous, then $h$ is continuous on $[-1, 0]$. Similarly, since $h = g$ on $(0, 1]$ and $g$ is continuous, then $h$ is continuous on $(0, 1]$. 
	
	It remains only to show that $h$ is continuous at $x = 0$.
	
	Let $\epsilon_0 > 0$. By continuity of $f$ $\exists \delta_0$ such that if $|x| < \delta_0$, then $|f(x) - f(0)| < \epsilon_0$. Similarly, let $\epsilon_1 >0$. Then by continuity of $g$, $\exists \delta_1 > 0$ such that if $|x| < \delta_1$, then $|g(x) - g(0)| < \epsilon_1$.
	
	Let $x \in [-1, 1]$ and let $\epsilon = \min\{\epsilon_0, \epsilon_1\}$. Choose $\delta = \min\{\delta_0, \delta_1\}$. If $|x| < \delta$, then $|x| < \delta_0$ and $|x| < \delta_1$. If $x \leq 0$, then we have
	\begin{equation}
		|h(x) - h(0)| = |f(x) - f(0)| < \epsilon_0 \leq \epsilon.
	\end{equation}
	Similarly if $x > 0$, then we have
	\begin{equation}
		|h(x) - h(0)| = |g(x) - g(0)| < \epsilon_1 \leq \epsilon.
	\end{equation}
	Thus, $h(x)$ is continuous at $x=0$, so we conclude $h$ is continuous.
\end{proof}
%%%%%%%%%%%%%%%%%%%%%%%%%%%%%%%%%%%%%%%%%%%%%%%%%%%%%%%%%%%%%%%%%%%%%%%%%%%%%%%%%%%%%%%%%%%%%%%%%%%%%%%%%%%%%%%%%%%%%%%%%%%%%%%%%%%%%%%%%%%
\end{document}