\documentclass{article}
\usepackage[utf8]{inputenc}
\usepackage[english]{babel}
\usepackage[]{amsthm} %lets us use \begin{proof}
\usepackage[]{amssymb} %gives us the character \varnothing
\usepackage[]{amsmath}
\usepackage[parfill]{parskip} %avoid indent when skipping lines

\title{18.100A Assignment 2}
\author{Octavio Vega}
\date\today

\begin{document}
\maketitle
	
\section*{Problem 1}
\begin{proof}
	(By contradiction).
	
	Suppose instead that $xy \leq xz$. Then
	
	$\implies xy - xz \leq 0$
	
	$\implies x(y-z) \leq 0$.
	
	Since $x<0$ by assumption, it must then be true that $y-z\geq 0$. But then
	
	$\implies y \geq z \quad \Rightarrow\Leftarrow$,
	
	which is a contradiction since we assumed that $y<z$. Thus, $xy>xz$.
\end{proof}

\section*{Problem 2}
\subsection*{(a)}
\begin{proof}
	We want to show that $\exists b\in S$ such that $\forall a\in A$, $a\leq b$.

	Since $S$ is ordered, then for every $x,y\in S$, we have that either $x<y$, $x>y$, or $x=y$. But since $A\subset S$, then $\forall a\in A$, $a\in A \implies a\in S$. 
	
	$\implies \forall a,b\in A$, either $a<b$, $a>b$, or $a=b$. 
	
	So $A$ is also ordered. Since $A$ is finite, then $\exists a_0 \in A$ such that $\forall a \in A$, $a_0 \geq a$. 
	
	Thus, $A$ is bounded.
\end{proof}

\subsection*{(b)}
\begin{proof}
	(By contradiction).
	
	Assuming $A$ is finite, suppose instead that there is no maximal element in $A$. Choose an element $a_1 \in A$. Then, since $a_1$ is not the maximum, $\exists a_2 \in A$ such that $a_1 < a_2$. But $a_2$ is also not the maximum of $A$, so $\exists a_3 \in A$ such that $a_2 < a_3$. Continuing in this manner, we find an increasing sequence $\{a_n\}_{n\in\mathbb{N}}$ of elements of $A$, i.e. such that 
	\begin{equation}
		a_1 < a_2 < \cdots < a_n < a_{n+1} < \cdots.
	\end{equation}
	But because this sequence is infinite and contained in $A$, this contradicts the assumption that $A$ is finite. Thus, there must exist a maximal element in $A$. 
	
	To show that there exists a minimum element, we recreate the same argument from above where instead, supposing that there is no minimal element, we demonstrate that we can construct an infinite decreasing sequence \\$\cdots a_n < \cdots <a_2 < a_1$ of elements of $A$, once again arriving at a contradiction.
	
	Therfore, both $\inf{A}$ and $\sup{A}$ exist in $A$.
\end{proof}
	
\end{document}