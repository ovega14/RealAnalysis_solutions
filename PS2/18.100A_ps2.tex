\documentclass{article}
\usepackage[utf8]{inputenc}
\usepackage[english]{babel}
\usepackage[]{amsthm} %lets us use \begin{proof}
\usepackage[]{amssymb} %gives us the character \varnothing
\usepackage[]{amsmath}
\usepackage[parfill]{parskip} %avoid indent when skipping lines

\title{18.100A Assignment 2}
\author{Octavio Vega}
\date\today

\begin{document}
\maketitle
	
\section*{Problem 1}
\begin{proof}
	(By contradiction).
	
	Suppose instead that $xy \leq xz$. Then
	
	$\implies xy - xz \leq 0$
	
	$\implies x(y-z) \leq 0$.
	
	Since $x<0$ by assumption, it must then be true that $y-z\geq 0$. But then
	
	$\implies y \geq z \quad \Rightarrow\Leftarrow$,
	
	which is a contradiction since we assumed that $y<z$. Thus, $xy>xz$.
\end{proof}

\section*{Problem 2}
\subsection*{(a)}
\begin{proof}
	We want to show that $\exists b\in S$ such that $\forall a\in A$, $a\leq b$.

	Since $S$ is ordered, then for every $x,y\in S$, we have that either $x<y$, $x>y$, or $x=y$. But since $A\subset S$, then $\forall a\in A$, $a\in A \implies a\in S$. 
	
	$\implies \forall a,b\in A$, either $a<b$, $a>b$, or $a=b$. 
	
	So $A$ is also ordered. Since $A$ is finite, then $\exists a_0 \in A$ such that $\forall a \in A$, $a_0 \geq a$. 
	
	Thus, $A$ is bounded.
\end{proof}

\subsection*{(b)}
\begin{proof}
	(By contradiction).
	
	Assuming $A$ is finite, suppose instead that there is no maximal element in $A$. Choose an element $a_1 \in A$. Then, since $a_1$ is not the maximum, $\exists a_2 \in A$ such that $a_1 < a_2$. But $a_2$ is also not the maximum of $A$, so $\exists a_3 \in A$ such that $a_2 < a_3$. Continuing in this manner, we find an increasing sequence $\{a_n\}_{n\in\mathbb{N}}$ of elements of $A$, i.e. such that 
	\begin{equation}
		a_1 < a_2 < \cdots < a_n < a_{n+1} < \cdots.
	\end{equation}
	But because this sequence is infinite and contained in $A$, this contradicts the assumption that $A$ is finite. Thus, there must exist a maximal element in $A$. 
	
	To show that there exists a minimum element, we recreate the same argument from above where instead, supposing that there is no minimal element, we demonstrate that we can construct an infinite decreasing sequence \\$\cdots a_n < \cdots <a_2 < a_1$ of elements of $A$, once again arriving at a contradiction.
	
	Therfore, both $\inf{A}$ and $\sup{A}$ exist in $A$.
\end{proof}

\section*{Problem 3}
\begin{proof}
	Since $b$ is an upper bound for $A$, then $\forall a\in A$, $a\leq b$. 
	
	Suppose $b \neq \sup{A}$. Then $\exists$ some other element $c \in A$ such that $c = \sup{A}$, since by problem 2, $A$ must have a supremum because it is finite and a subset of an ordered set. But since $b \in A$, then $b \leq c$. 
	
	However, we assumed that $b$ is an upper bound for $A$, so since $c \in A$, this imples that $b \geq c$. Thus we have that $b \leq c$ and $b \geq c$, so it must hold that $b=c$.
	
	Therefore $b = \sup{A}$, as desired. 
\end{proof}

\section*{Problem 4}
\begin{proof}
	Suppose $\sup{A}\notin A$, and let $x_0 \in A$. Towards a contradiction, suppose that $\forall x \in A$, $x \leq x_0$. Then $x_0$ is an upper bound for $A$.
	
	Since $x_0 \in A$, then by problem 3, $x_0 = \sup{A}$. But we assumed $\sup{A} \notin A$, so contradiction.
	
	So for any $x_0 \in A$, $\exists x_1$ such that $x_1 > x_0$. We can repeat the logic above to show that this holds for arbitrary $x_i \in A$, since no $x_i$ can both be an upper bound for $A$ while also being contained in $A$. Thus $\forall x_i \in A$, $\exists x_{i+1}$ such that $x_i < x_{i+1}$. 
	
	So we obtain an infinite decreasing sequence $\{x_i\}_{i \in \mathbb{N}}$ of elements of $A$. Hence, these elements form a countably infinite subset of $A$, since $|\{x_i | i \in \mathbb{N}\}| = |\mathbb{N}|$.
	
	Therefore, $A$ does indeed contain a countably infinite subset.
\end{proof}

\section*{Problem 5}
\subsection*{(a)}
\begin{proof}
	(Arithmetic Mean - Geometric Mean Inequality)
	
	Consider the difference $\sqrt{x} - \sqrt{y}$. We compute:
	\begin{align}
		0 \leq \left(\sqrt{x} - \sqrt{y}\right)^2 &= x + y -2 \sqrt{xy} \\
		\implies 2\sqrt{xy} &\leq x + y \label{amgm}
	\end{align} 
	Dividing both sides of \eqref{amgm} by 2, we obtain the desired result: $\sqrt{xy} \leq \frac{x + y}{2}$.
\end{proof}
\subsection{(b)}

\end{document}