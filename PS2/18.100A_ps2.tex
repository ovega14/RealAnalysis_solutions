\documentclass{article}
\usepackage[utf8]{inputenc}
\usepackage[english]{babel}
\usepackage[]{amsthm} %lets us use \begin{proof}
\usepackage[]{amssymb} %gives us the character \varnothing
\usepackage[]{amsmath}
\usepackage[parfill]{parskip} %avoid indent when skipping lines

\title{18.100A Assignment 2}
\author{Octavio Vega}
\date\today

\begin{document}
\maketitle
	
\section*{Problem 1}
\begin{proof}
	(By contradiction).
	
	Suppose instead that $xy \leq xz$. Then
	
	$\implies xy - xz \leq 0$
	
	$\implies x(y-z) \leq 0$.
	
	Since $x<0$ by assumption, it must then be true that $y-z\geq 0$. But then
	
	$\implies y \geq z \quad \Rightarrow\Leftarrow$,
	
	which is a contradiction since we assumed that $y<z$. Thus, $xy>xz$.
\end{proof}

\section*{Problem 2}
\subsection*{(a)}
\begin{proof}
	We want to show that $\exists b\in S$ such that $\forall a\in A$, $a\leq b$.

	Since $S$ is ordered, then for every $x,y\in S$, we have that either $x<y$, $x>y$, or $x=y$. But since $A\subset S$, then $\forall a\in A$, $a\in A \implies a\in S$. 
	
	$\implies \forall a,b\in A$, either $a<b$, $a>b$, or $a=b$. 
	
	So $A$ is also ordered. Since $A$ is finite, then $\exists a_0 \in A$ such that $\forall a \in A$, $a_0 \geq a$. 
	
	Thus, $A$ is bounded.
\end{proof}

\subsection*{(b)}
\begin{proof}
	(By contradiction).
	
	Assuming $A$ is finite, suppose instead that there is no maximal element in $A$. Choose an element $a_1 \in A$. Then, since $a_1$ is not the maximum, $\exists a_2 \in A$ such that $a_1 < a_2$. But $a_2$ is also not the maximum of $A$, so $\exists a_3 \in A$ such that $a_2 < a_3$. Continuing in this manner, we find an increasing sequence $\{a_n\}_{n\in\mathbb{N}}$ of elements of $A$, i.e. such that 
	\begin{equation}
		a_1 < a_2 < \cdots < a_n < a_{n+1} < \cdots.
	\end{equation}
	But because this sequence is infinite and contained in $A$, this contradicts the assumption that $A$ is finite. Thus, there must exist a maximal element in $A$. 
	
	To show that there exists a minimum element, we recreate the same argument from above where instead, supposing that there is no minimal element, we demonstrate that we can construct an infinite decreasing sequence \\$\cdots a_n < \cdots <a_2 < a_1$ of elements of $A$, once again arriving at a contradiction.
	
	Therfore, both $\inf{A}$ and $\sup{A}$ exist in $A$.
\end{proof}

\section*{Problem 3}
\begin{proof}
	Since $b$ is an upper bound for $A$, then $\forall a\in A$, $a\leq b$. 
	
	Suppose $b \neq \sup{A}$. Then $\exists$ some other element $c \in A$ such that $c = \sup{A}$, since by problem 2, $A$ must have a supremum because it is finite and a subset of an ordered set. But since $b \in A$, then $b \leq c$. 
	
	However, we assumed that $b$ is an upper bound for $A$, so since $c \in A$, this imples that $b \geq c$. Thus we have that $b \leq c$ and $b \geq c$, so it must hold that $b=c$.
	
	Therefore $b = \sup{A}$, as desired. 
\end{proof}

\section*{Problem 4}
\begin{proof}
	Suppose $\sup{A}\notin A$, and let $x_0 \in A$. Towards a contradiction, suppose that $\forall x \in A$, $x \leq x_0$. Then $x_0$ is an upper bound for $A$.
	
	Since $x_0 \in A$, then by problem 3, $x_0 = \sup{A}$. But we assumed $\sup{A} \notin A$, so contradiction.
	
	So for any $x_0 \in A$, $\exists x_1$ such that $x_1 > x_0$. We can repeat the logic above to show that this holds for arbitrary $x_i \in A$, since no $x_i$ can both be an upper bound for $A$ while also being contained in $A$. Thus $\forall x_i \in A$, $\exists x_{i+1}$ such that $x_i < x_{i+1}$. 
	
	So we obtain an infinite decreasing sequence $\{x_i\}_{i \in \mathbb{N}}$ of elements of $A$. Hence, these elements form a countably infinite subset of $A$, since $|\{x_i | i \in \mathbb{N}\}| = |\mathbb{N}|$.
	
	Therefore, $A$ does indeed contain a countably infinite subset.
\end{proof}

\section*{Problem 5}
\subsection*{(a)}
\begin{proof}
	(Arithmetic Mean - Geometric Mean Inequality)
	
	Consider the difference $\sqrt{x} - \sqrt{y}$. We compute:
	\begin{align}
		0 \leq \left(\sqrt{x} - \sqrt{y}\right)^2 &= x + y -2 \sqrt{xy} \\
		\implies 2\sqrt{xy} &\leq x + y \label{amgm}
	\end{align} 
	Dividing both sides of \eqref{amgm} by 2, we obtain the desired result: $\sqrt{xy} \leq \frac{x + y}{2}$.
\end{proof}
\subsection*{(b)}
\begin{proof}
	($\Rightarrow$) Suppose $\sqrt{xy} = \frac{x+y}{2}$. Then
	\begin{align}
		&\implies 2\sqrt{xy} - x - y = 0 \\
		&\implies x + y - 2\sqrt{xy} = 0 \\
		&\implies \left(\sqrt{x} - \sqrt{y}\right)^2 = 0 \\
		&\implies \sqrt{x} - \sqrt{y} = 0 \\
		&\implies \sqrt{x} = \sqrt{y} \\
		&\implies x = y.
	\end{align}
	($\Leftarrow$) Suppose $x=y$. Then
	\begin{align}
		\frac{x+y}{2} &= \frac{x+x}{2} = x \\
		&= \sqrt{x}\cdot \sqrt{x} = \sqrt{x}\cdot \sqrt{y}\\
		&= \sqrt{xy}.
	\end{align}
	Thus, $\frac{x+y}{2}=\sqrt{xy} \iff x=y$.
\end{proof}

\section*{Problem 6}
\subsection*{(a)}
\begin{proof}
	Since $A$ is bounded, then $\exists a_0$ such that $\forall a \in A$, $a \geq a_0$. Similarly, since $B$ is bounded, then $\exists b_0$ such that $\forall b \in B$, $b \geq b_0$. 
	
	Let $c\in C$, i.e. let $a\in A$, $b\in B$, and take $c = a + b$. Then for all such $c$, 
	\begin{equation}
		c = a + b \geq a_0 + b \geq a_0 + b_0
	\end{equation}
	Hence, the set $C$ is bounded below. By the same logic, we can show that $\forall c\in C$, $c<a_1 + b_1$ where $a_1$ and $b_1$ are upper bounds for $A$ and $B$, respectively. So $C$ is also bounded above.
	
	Therefore, $C$ is bounded.
\end{proof}

\subsection*{(b)}
\begin{proof}
	Let $\epsilon > 0$. By definition of supremum, $\exists x \in A$, $y\in B$ such that $x + \frac{\epsilon}{2} > \sup{A}$ and $y + \frac{\epsilon}{2} > \sup{B}$. Then
	\begin{align}
		x + y + \epsilon &= \left(x+\frac{\epsilon}{2}\right) + \left(y+\frac{\epsilon}{2}\right) \\
		&> \sup{A} + \sup{B}.
	\end{align} 
	So for every $\epsilon >0$, $\exists c = x+y \in C$ such that 
	\begin{equation}
		c + \epsilon > \sup{A} + \sup{B}. 
	\end{equation}
	Then we conclude that $\sup{C} = \sup{A} + \sup{B}$. 
\end{proof}

\subsection*{(c)}
\begin{proof}
	Let $\epsilon > 0$. By definition of infimum, $\exists x \in A$, $y\in B$ such that $x - \frac{\epsilon}{2} < \inf{A}$ and $y - \frac{\epsilon}{2} < \inf{B}$. Then
	\begin{align}
		x + y - \epsilon &= \left(x-\frac{\epsilon}{2}\right) + \left(y-\frac{\epsilon}{2}\right) \\
		&< \inf{A} + \inf{B}.
	\end{align} 
	So for every $\epsilon >0$, $\exists c = x+y \in C$ such that 
	\begin{equation}
		c - \epsilon < \inf{A} + \inf{B}. 
	\end{equation}
	Then we conclude that $\inf{C} = \inf{A} + \inf{B}$.
\end{proof}

\section*{Problem 7}
\subsection*{(a)}
\begin{proof}
	Let $E = \{x\in\mathbb{R} \: | \: x>0 \textrm{ and } x^3 < 2\}$. So $\forall x \in E$, $x^3 < 2$. Then
	\begin{align}
		&\implies x^3 - 2 < 0 \\
		&\implies \left(x-2^{\frac{1}{3}}\right)\left(x^2 + 2^{\frac{1}{3}}x + 2^{\frac{2}{3}}\right) < 0.
	\end{align}
	But $x>0 \implies x^2 + 2^{\frac{1}{3}}x + 2^{\frac{2}{3}} > 0$, so $\forall x \in E$, 
	\begin{align}
		&\implies x-2^{\frac{1}{3}} < 0 \\
		&\implies x < \sqrt[3]{2}.
	\end{align}
	Thus, $E$ is bounded above.
\end{proof}

\subsection*{(b)}
\begin{proof}
	Let $r = \sup{E}$, which exists by part (a) and problem 2 (b). We first show that $r>0$. 
	
	By definition of the supremum and the set $E$, $\forall x \in E$, $x\leq r$ and $x>0$. Thus $0<x\leq r$, so $r$ is indeed positive.
	
	Next we show that $r^3 = 2$. 
	
	Suppose, toward a contradiction, that $r > \sqrt[3]{2}$. Since $r = \sup{E}$, then $x\leq r$ for any $x \in E$. But since $\sqrt[3]{2}$ is an upper bound for $E$, then $x < \sqrt[3]{2}$ $\forall x\in E$. So 
	\begin{align}
		&\implies x < \sqrt[3]{2} < r \\
		&\implies r \neq \sup{E}, \quad \Rightarrow\Leftarrow.
	\end{align}
	This is a contradiction, since we assumed that $r = \sup{E}$. Thus $r \leq \sqrt[3]{2} \implies r^3 \leq 2$.
	
	Now suppose instead, toward another contradiction, that $r<\sqrt[3]{2}$. Then
	\begin{equation}
		r^3 < \left(\sqrt[3]{2}\right)^3 = 2.
	\end{equation}
	But then $r\in E$, and $\sqrt[3]{2}>r$ which implies that $r \neq \sup{E}$, another contradiction. Thus $r \geq \sqrt[3]{2}$. So we have shown that both $r \leq \sqrt[3]{2}$ and $r \geq \sqrt[3]{2}$. Then we conclude $r=\sqrt[3]{2}$.
	
	Hence, $r^3 = 2$.
\end{proof}

\end{document}