\documentclass{article}
\usepackage[utf8]{inputenc}
\usepackage[english]{babel}
\usepackage[]{amsthm} %lets us use \begin{proof}
\usepackage[]{amssymb} %gives us the character \varnothing
\usepackage[]{amsmath}
\usepackage[parfill]{parskip} %avoid indent when skipping lines

\title{18.100A Assignment 1}
\author{Octavio Vega}
\date\today

\begin{document}
\maketitle
	
\section*{Problem 1}
\subsection*{(a)}
\begin{proof}
	We will show that each set is a subset of the other to prove equality. Let $S=A\cap (B\cup C)$ and $T=(A\cup B)\cap (A\cup C)$. 
	
	Let $x \in S$. Then $x\in A$ and $x\in B\cup C$, 
	
	$\implies$ $x\in A$ and $x\in B$, or $x\in A$ and $x\in C$
	
	$\implies$ $x\in A\cap B$ or $x\in A\cap C$
	
	$\implies$ $x\in (A\cap B) \cup (A\cap C) = T$.
	
	Thus $x\in S \implies x\in T$, so $S\subseteq T$. Now let $x\in T$. Then $x\in (A\cap B) \cup (A\cap C)$,
	
	$\implies$ $x\in A\cap B$ or $x\in A \cap C$
	
	$\implies$ $x\in A$, and $x\in B$ or $C$
	
	$\implies$ $x\in A \cap (B\cup C) = S$. 
	
	Thus $x\in T \implies x\in S$, so $T\subseteq S$, which means $S=T$.
	
	Hence, $A\cap (B\cup C) = (A\cap B) \cup (A\cap C)$.
	
\end{proof}

\subsection*{(b)}
\begin{proof}
	We proceed as in (a). Let $S = A \cup (B\cap C)$ and $T = (A\cup B) \cap (A\cup C)$. 
	
	First let $x\in S$. Then $x\in A$ or $x\in B\cap C$,
	
	$\implies$ $x\in A$ or $x\in B$, and $x\in A$ or $x\in C$
	
	$\implies$ $x\in (A\cup B)\cap(A\cup C) = T$.
	
	Thus $x\in S \implies x\in T$, so $S\subseteq T$. Now let $x\in T$. Then $x\in A\cup B$ and $x\in A\cup C$. If $x\in A$, then the requirement is satisfied immediately, regardless of whether $x$ is in $B$ or $C$. Otherwise, if $x\notin A$, then $x\in B$ and $x\in C$ must be true. So $x\in A$, or $x\in B$ and $x\in C$
	
	$\implies$ $x\in A\cup(B\cap C) = S$. 
	
	Thus $x\in T \implies x\in S$, so $T \subseteq S$, which means $S = T$. 
	
	Hence, $A\cup(B\cap C) = (A\cup B) \cap (A\cup C)$.
\end{proof}

\section*{Problem 2}
\begin{proof}
	(By induction). 
	
	The inductive hypothesis $P(n)$ is that, for $n\in\mathbb{N}$, $n<2^n$.
	
	(Base case): $1 < 2^1$, so $P(1)$ is true. 
	
	(Inductive step): Assume $P(n)$ is true for $n=m$, i.e. that $m < 2^m$ holds for $m\in\mathbb{N}$. Then
	\begin{align}
		2^{m+1} = 2 \cdot 2^m &> 2m \
		&=m + m > m + 1, \quad \textrm{since $m>1$} \
		\implies 2^{m+1} &> m + 1.
	\end{align}
	So $P(m)\implies P(m+1)$, so $P(n)$ is true for all $n\in\mathbb{N}$. 
	
	Thus, $2^n > n$ $\forall n\in\mathbb{N}$.
\end{proof}
\end{document}