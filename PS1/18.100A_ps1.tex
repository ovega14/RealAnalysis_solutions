\documentclass{article}
\usepackage[utf8]{inputenc}
\usepackage[english]{babel}
\usepackage[]{amsthm} %lets us use \begin{proof}
\usepackage[]{amssymb} %gives us the character \varnothing
\usepackage[]{amsmath}
\usepackage[parfill]{parskip} %avoid indent when skipping lines

\title{18.100A Assignment 1}
\author{Octavio Vega}
\date\today

\begin{document}
\maketitle
	
\section*{Problem 1}
\subsection*{(a)}
\begin{proof}
	We will show that each set is a subset of the other to prove equality. Let $S=A\cap (B\cup C)$ and $T=(A\cup B)\cap (A\cup C)$. 
	
	Let $x \in S$. Then $x\in A$ and $x\in B\cup C$, 
	
	$\implies$ $x\in A$ and $x\in B$, or $x\in A$ and $x\in C$
	
	$\implies$ $x\in A\cap B$ or $x\in A\cap C$
	
	$\implies$ $x\in (A\cap B) \cup (A\cap C) = T$.
	
	Thus $x\in S \implies x\in T$, so $S\subseteq T$. Now let $x\in T$. Then $x\in (A\cap B) \cup (A\cap C)$,
	
	$\implies$ $x\in A\cap B$ or $x\in A \cap C$
	
	$\implies$ $x\in A$, and $x\in B$ or $C$
	
	$\implies$ $x\in A \cap (B\cup C) = S$. 
	
	Thus $x\in T \implies x\in S$, so $T\subseteq S$, which means $S=T$.
	
	Hence, $A\cap (B\cup C) = (A\cap B) \cup (A\cap C)$.
	
\end{proof}

\subsection*{(b)}
\begin{proof}
	We proceed as in (a). Let $S = A \cup (B\cap C)$ and $T = (A\cup B) \cap (A\cup C)$. 
	
	First let $x\in S$. Then $x\in A$ or $x\in B\cap C$,
	
	$\implies$ $x\in A$ or $x\in B$, and $x\in A$ or $x\in C$
	
	$\implies$ $x\in (A\cup B)\cap(A\cup C) = T$.
	
	Thus $x\in S \implies x\in T$, so $S\subseteq T$. Now let $x\in T$. Then $x\in A\cup B$ and $x\in A\cup C$. If $x\in A$, then the requirement is satisfied immediately, regardless of whether $x$ is in $B$ or $C$. Otherwise, if $x\notin A$, then $x\in B$ and $x\in C$ must be true. So $x\in A$, or $x\in B$ and $x\in C$
	
	$\implies$ $x\in A\cup(B\cap C) = S$. 
	
	Thus $x\in T \implies x\in S$, so $T \subseteq S$, which means $S = T$. 
	
	Hence, $A\cup(B\cap C) = (A\cup B) \cap (A\cup C)$.
\end{proof}

\section*{Problem 2}
\begin{proof}
	(By induction). 
	
	The inductive hypothesis $P(n)$ is that, for $n\in\mathbb{N}$, $n<2^n$.
	
	(Base case): $1 < 2^1$, so $P(1)$ is true. 
	
	(Inductive step): Assume $P(n)$ is true for $n=m$, i.e. that $m < 2^m$ holds for $m\in\mathbb{N}$. Then
	\begin{align}
		2^{m+1} = 2 \cdot 2^m &> 2m \\
		&=m + m > m + 1, \quad \textrm{since $m>1$} \\
		\implies 2^{m+1} &> m + 1.
	\end{align}
	So $P(m)\implies P(m+1)$, which means $P(n)$ is true for all $n\in\mathbb{N}$. 
	
	Thus, $\forall n\in\mathbb{N}$, $2^n > n$.
\end{proof}

\section*{Problem 3}
\begin{proof}
	Let $A$ be a finite set such that $|A|=n$. We form the power set $\mathcal{P}(A)$ by creating the set of all possible subsets of $A$. Hence $|\mathcal{P}(A)|$ is equivalent to the number of possible subsets of $A$, which we compute by summing over the number of combinations that can be created by choosing elements of $A$, in succession from choosing no elements (the empty set $\emptyset$) to choosing all elements (the full set $A$). Thus
	\begin{equation}
		|\mathcal{P}(A)| = \sum_{k=0}^n  \begin{pmatrix} n \\ k \end{pmatrix}.
	\end{equation}
	By the binomial expansion theorem,
	\begin{equation}\label{binom}
		\sum_{k=0}^n \begin{pmatrix} n \\ k \end{pmatrix} p^k q^{n-k} = (p+q)^n.
	\end{equation}
	Substituting $p=q=1$ into \eqref{binom}, we arrive at the desired result,
	\begin{equation}
		|\mathcal{P}(A)| = (1 + 1)^n = 2^n.
	\end{equation}
\end{proof}

\section*{Problem 4}
\begin{proof}
	(By induction). 
	
	$P(n)$ is the hypothesis that for $n\in\mathbb{N}$, $n^3 + 5n$ is divisible by $6$.
	
	(Base case): $1^3 + 5\cdot1 = 1 + 5 = 6$ is divisible by 6, so $P(1)$ is true. 
	
	(Inductive step): Assume $P(m)$ holds, i.e. 6 divides $m^3 + 5m$. Then
	\begin{align}
		(m+1)^3 + 5m &= m^3 + 3m^2 + 3m + 1 + 5m + 1 \\
		&= m^3 + 5m + 6 + 3m(m+1), 
	\end{align}
	where by the inductive hypothesis $m^3 +5m + 6$ is divisible by 6 and $3m(m+1)$ is also divisible by 6 because it is divisible by both 3 and 2. So, their sum $(m+1)^3 + 5(m+1)$ must also be divisible by 6, which means $P(m)\implies P(m+1)$.
	
	Thus, $\forall n\in\mathbb{N}$, $n^3 + 5n$ is divisible by 6. 
\end{proof}

\section*{Problem 5}
\begin{proof}
	Let $A_1, A_2, A_3, ..., A_n, A_{n+1}, ...$ be finite sets, and let there be infinitely many of them. We wish to find an example where $|\bigcup\limits_{i=1}^{\infty} A_i| = \infty$. 
	
	To satisfy this, choose $A_{i} = \{i\}$. Then $A_1 \cup A_2 \cup A_3 \cdots = \{1, 2, 3, ... \} = \mathbb{N}$. So their union is not a finite set, as desired. 
\end{proof}

\section*{Problem 6}
\subsection*{(a)}
Consider $q=\frac{4}{15}$. Then 
\begin{equation}
	\frac{4}{15} = \frac{2\cdot2}{3\cdot5} = \frac{2^2}{3\cdot5}.
\end{equation}
So we compute
\begin{equation}
	f(\frac{4}{15}) = 2^{2\cdot2} \cdot 3^{2-1}\cdot5^{2-1} = 2^4\cdot3\cdot5.
\end{equation}
Hence, $f(\frac{4}{15})=240$.

Now suppose $f(q) = 108.$ Then we write
\begin{equation}
	108 = 2\cdot54 = 2\cdot2\cdot27 = 2^2 \cdot 3^3 = p_1^{2r_1}q_1^{2s_1 - 1},
\end{equation}
so we identify $r_1 = 1$ and $s_1=2$. Thus, 
\begin{equation}
	q = \frac{p_1^{r_1}}{q_1^{s_1}} = \frac{2}{3^2} = \frac{2}{9}.
\end{equation}
\end{document}